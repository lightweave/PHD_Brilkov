\chapter{Обзор Литературы} \label{chapt1}

\section{Радиационная обстановка на высокоширотных околоземных орбитах. Вопросы, требующие детального исследования.}

Исследования радиационной обстановки в космическом пространстве связано с началом полетов автоматических аппаратов и человека в космос.  Широкое распространение технологий, связанных с использованием космической техники, а также непрерывные пребывание человека в космическом пространстве во время миссий на космических станциях МИР и МКС позволило выявить ряд опасностей космических полетов, среди которых особое внимание следует уделить радиационной опасности [Логачев Ю.И. 2007].

