\appendix
%% Правка оформления ссылок на приложения:
%http://tex.stackexchange.com/questions/56839/chaptername-is-used-even-for-appendix-chapters-in-toc
%http://tex.stackexchange.com/questions/59349/table-of-contents-with-chapter-and-appendix
%% требует двойной компиляции
\addtocontents{toc}{\def\protect\cftchappresnum{\appendixname{} }%
\setlength{\cftchapnumwidth}{\widthof{\cftchapfont\appendixname~Ш\cftchapaftersnum}}%
}
%% Оформление заголовков приложений ближе к ГОСТ:
\sectionformat{\chapter}[display]{% Параметры заголовков разделов в тексте
    label=\chaptertitlename\ \thechapter,% (ГОСТ Р 2.105, 4.3.6)
    labelsep=20pt,
}
\renewcommand\thechapter{\Asbuk{chapter}} % Чтобы приложения русскими буквами нумеровались



\chapter{Структура данных измерений ДЭПРОН} \label{Appendixa}

\section{Блок данных ДЭПРОН A}




{\small Данные (RECORD)Время Аппаратный счетчик детектора }{\small 1}{\small Счет детектора }{\small 1}{\small Счет детектора 2Счет совпадениймесяцденьчасмин1 байт1 байт1 байт1 байт2 байта2 байта2 байта2 байта}


Продолжение:


{\small Данные (RECORD)Нейтронный счетчик 1Нейтронный счетчик 2Доза по первому детекторуДоза по второму детекторуДоза совпаденийМассивы секундной динамики2 байта2 байта}{\small 4}{\small  байта}{\small 4}{\small  байта}{\small 4}{\small  байта480 байт}





Блок массивов секундной динамики содержат шестьдесят массивов, содержащих приращения значений счетчиков за секунду, сжатых алгоритмом логарифмического сжатия.





{\small Массив секундной динамикиАппаратный счетчик детектора }{\small 1}{\small Счет детектора }{\small 1}{\small Счет детектора 2Счет совпадений}1 байт1 байт1 байт1 байт


Продолжение:


{\small Массив секундной динамикиНейтронный счетчик 1Нейтронный счетчик 2Доза по первому детекторуДоза по второму детектору}1 байт1 байт1 байт1 байт









\section{Блок данных ДЭПРОН S}




{\footnotesize Данные (RECORD)Время Спектр детектора 1Счетчик детектора 1Спектр детектора 2Счетчик детектора 2Спектр совпадений 1Счетчик совпаденийСпектр совпадениймесяцденьчасмин1 байт1 байт1 байт1 байт124 байта4 байта124 байта4 байта124 байта4 байта124 байта}


Каждый передаваемый спектр содержит число зарегистрированных импульсов, попадающих в соответствующий энергетический канал детектора. Количество энергетических диапазонов 62, верхние границы каналов выбраны с помощью алгоритма логарифмического преобразования номера канала.


\textbf{{\small Канал №Код АЦПКанал №Код АЦПКанал №Код АЦПКанал №Код АЦП}}{\small 216181603464050256032419176357045128164322019236768523072540212083783253332864822224388965435847562324039960553840864242564010245640969722528841115257460810802632042128058512011882735243140859563212962838444153660614413104294164516646166561411230448461792627168151203148047192063768016128325124820481714433576492304}


Таким образом, массивы спектров состоят из 62 четырехбайтных целых значений, порядок которых соответствует приведенному набору каналов.




\section{Блок данных ДЭПРОН H}


{\footnotesize Данные (RECORD)Время Индекс массива высоких амплитуд(по умолчанию значение 63)Массивы данных по высоким амплитудаммесяцдень1 байт1 байт2 байта63 массива  по 8 байт}



Структура записи в массивы данных по высоким амплитудам:



{\footnotesize Массив данных по высоким амплитудам Код АЦП 1Код АЦП 2Время событияКод таймерасекундаминутачас2 байта2 байта1 байт1 байт1 байт1 байт}


\begin{flushleft}
	
\end{flushleft}


\begin{flushleft}
	\newpage
	
\end{flushleft}


\section{Блок данных ДЭПРОН N}

\begin{flushleft}
	
\end{flushleft}


{\footnotesize Данные (RECORD)Массивы данных по нейтронным всплескам127 массивов  по 4 байт}


\begin{flushleft}
	Структура записи в массив данных по нейтронным всплескам:
\end{flushleft}


3231302928272625242322212019181716{\small Время дня в секундах}


\begin{flushleft}
	
\end{flushleft}


151413121110987654321{\small НДКоличество тиков таймера после предыдущего нейтронного импульса}


\begin{flushleft}
	
\end{flushleft}


\begin{flushleft}
	НД -- номер сработавшего нейтронного детектора:
\end{flushleft}


\begin{flushleft}
	01 -- срабатывание первого детектора
\end{flushleft}


\begin{flushleft}
	10 -- срабатывание второго детектора
\end{flushleft}


\begin{flushleft}
	11 -- срабатывание обоих детекторов
\end{flushleft}




\section{Блок данных ДЭПРОН Т}

В общей структуре сообщения для данного блока данных не выдается длина сообщения \texttt{LEN}, вместо этого передается {`}\ensuremath{\backslash}0'.(Исправлено 27.02.2013)


Данный блок данных генерируется в ответ на пришедшее по каналу CAN от БИ командное сообщение, либо по мере заполнения выходного буфера при работе в режиме отладки прибора ДЭПРОН.



\section{Команды прибора ДЭПРОН}



Для управления работой прибора ДЭПРОН предусмотрено 6 типов команд:

\begin{itemize}
	\item 	Сброс настроек к заводским параметрам

	\item Увеличение временного диапазона для нейтронных последовательностей

	\item Уменьшение временного диапазона для нейтронных последовательностей

	\item Увеличение полосы фильтра шумов протонных каналов

	\item Уменьшение полосы фильтра шумов протонных каналов

	\item Увеличение интервала времени сглаживания

\end{itemize}





\textbf{{\small №КомандаОписание команды}}{\small 1Clr}{\small сброс настроек к заводским параметрам2Tn+увеличение интервала между моментами регистрации нейтронов3Tn-уменьшение интервала между моментами регистрации нейтронов4Psnr+увеличение допустимого протонного фона5Psnr\_уменьшение допустимого протонного фона6Alpha+увеличение интервала времени сглаживания}


Ответный массив информации от прибора ДЭПРОН


{\small Данные (RECORD)Время Номер команды породившей ответD}{\small  }{\small tickСчетчик принятых командТекущий фон потока протонов (}{\small Pronon}{\small \_}{\small Fon}{\small )PsnrAlfaмесденьчасминсек1 байт1 байт1 байт1 байт1 байт4 байта}{\small 4}{\small  байта4 байта}{\small 4}{\small  байта}{\small 4}{\small  байта}{\small 4}{\small  байта}


\begin{flushleft}
	Где:
\end{flushleft}


\begin{flushleft}
	D tick - интервал времени меньше которого считается, что идет один нейтронный всплеск, мсек/6
\end{flushleft}


Psnr - максимальный уровень протонного фона, выше которого нейтронные всплески не регистрируются.


Alfa - константа экспоненциального сглаживания для расчета фона протонов





\section{Отладочные сообщения прибора ДЭПРОН}



Для проверки работы таймера высоких амплитуд и последовательности выполнения программного кода прибора ДЭПРОН существует возможность выдачи последовательностей измеренных временных промежутков, маркированных по названиям выполняемых блоков программного кода. Такая выдача происходит во время включения прибора ДЭПРОН в отладочном режиме, для этого необходима прошивка контроллера с объявленным макросом DEBUGTIME.


Измерение времени выполнения блоков происходит с помощью таймера прибора TC1. Информация записывается последовательно в блоки данных ДЭПРОН Т, каждый пакет имеет размер 4 байта, первые два байта отведены под идентификационные символы, последние два байта содержат накопленное значение таймера TC1, который настроен на работы с частотой 20 МГц.


{\small Данные (RECORD)Пакет таймера 1Пакет таймера 2\ldots{}\ldots{}\ldots{}\ldots{}Пакет таймера 2Символ 1Символ 2Значение таймера1 байт1 байт2 байт4 байт4 байт}





Таблица определения выполненных блоков по маркирующим символам пакета таймера.


\textbf{Символ 1Символ 2Выполненный блок}ADВыполнен блок External\_ADC\_Read\_DoubleE0x01Выполнен блок Ext\_Interrupt, вхождение блока P1E0x02Выполнен блок Ext\_Interrupt, вхождение блока P2E0x04Выполнен блок Ext\_Interrupt, вхождение блока N1E0x08Выполнен блок Ext\_Interrupt, вхождение блока N2E0x10Выполнен блок Ext\_Interrupt, вхождение блока T2D0Выполнен блок Detectors\_Handling, Detectors\_Flugs пустоеD1Выполнен блок Detectors\_Handling, до вызова External\_ADC\_Read\_DoubleD2Выполнен блок Detectors\_Handling, после вызова External\_ADC\_Read\_Double и до конца функцииN0Выполнен блок New\_Secunde\_Handler (вызов Data\_CAN\_Sending и Command\_Handler каждую секунду)N1Выполнен блок New\_Secunde\_Handler (сохранение текущей дозы каждую секунду)N2Выполнен блок New\_Secunde\_Handler (отправка накопленных за минуту данных и каждые пять минут спектра)


\chapter{Примеры вставки листингов программного кода} \label{AppendixA}

Для крупных листингов есть два способа. Первый красивый, но в нём могут быть проблемы с поддержкой кириллицы (у вас может встречаться в комментариях и
печатаемых сообщениях), он представлен на листинге~\ref{list:hwbeauty}.
%\renewcommand\FBbskip{-20pt} % если хотим притянуть что-то к плавающему окружению из floatrow
\begin{ListingEnv}[H]% буква H означает Here, ставим здесь,
    % элементы, которые нежелательно разрывать обычно не ставят
    % посреди страницы: вместо H используется t (top, сверху страницы),
    % или b (bottom) или p (page, на отдельной странице)
%    \captionsetup{format=tablenocaption}% должен стоять до самого caption
%    \thisfloatsetup{\capposition=top}%
    \caption{Программа “Hello, world” на \protect\cpp}
    % далее метка для ссылки:
    \label{list:hwbeauty}
    % окружение учитывает пробелы и табляции и приеняет их в сответсвии с настройкми
    \begin{lstlisting}[language={[ISO]C++}]
	#include <iostream>
	using namespace std;

	int main() //кириллица в комментариях при xelatex и lualatex имеет проблемы с пробелами
	{
		cout << "Hello, world" << endl; //latin letters in commentaries
		system("pause");
		return 0;
	}
    \end{lstlisting}
\end{ListingEnv}%
%Второй не такой красивый, но без ограничений (см.~листинг~\ref{list:hwplain}).
%\begin{ListingEnv}[H]
%    \begin{Verb}
%        
%        #include <iostream>
%        using namespace std;
%        
%        int main() //кириллица в комментариях
%        {
%            cout << "Привет, мир" << endl;
%        }
%    \end{Verb}
%    \caption{Программа “Hello, world” без подсветки}
%    \label{list:hwplain}
%\end{ListingEnv}
%
%Можно использовать первый для вставки небольших фрагментов
%внутри текста, а второй для вставки полного
%кода в приложении, если таковое имеется.
%
%Если нужно вставить совсем короткий пример кода (одна или две строки), то выделение  линейками и нумерация может смотреться чересчур громоздко. В таких случаях можно использовать окружения \texttt{lstlisting} или \texttt{Verb} без \texttt{ListingEnv}. Приведём такой пример с указанием языка программирования, отличного от заданного по умолчанию:
%\begin{lstlisting}[language=Haskell]
%fibs = 0 : 1 : zipWith (+) fibs (tail fibs)
%\end{lstlisting}
%Такое решение~--- со вставкой нумерованных листингов покрупнее
%и вставок без выделения для маленьких фрагментов~--- выбрано,
%например, в книге Эндрю Таненбаума и Тодда Остина по архитектуре
%%компьютера~\autocite{TanAus2013} (см.~рис.~\ref{fig:tan-aus}).
%
%Наконец, для оформления идентификаторов внутри строк
%(функция \lstinline{main} и тому подобное) используется
%\texttt{lstinline} или, самое простое, моноширинный текст
%(\texttt{\textbackslash texttt}).


Пример~\ref{list:internal3}, иллюстрирующий подключение переопределённого языка. Может быть полезным, если подсветка кода работает криво. Без дополнительного окружения, с подписью и ссылкой, реализованной встроенным средством.
\todo{вставить пример получения квантилей, подбор параметров и оптимизацию}
\begin{lstlisting}[language={Renhanced},caption={Пример листинга c подписью собственными средствами},label={list:internal3}]
## Caching the Inverse of a Matrix

## Matrix inversion is usually a costly computation and there may be some
## benefit to caching the inverse of a matrix rather than compute it repeatedly
## This is a pair of functions that cache the inverse of a matrix.

## makeCacheMatrix creates a special "matrix" object that can cache its inverse

makeCacheMatrix <- function(x = matrix()) {#кириллица в комментариях при xelatex b lualatex имеет проблемы с пробелами
    i <- NULL
    set <- function(y) {
        x <<- y
        i <<- NULL
    }
    get <- function() x
    setSolved <- function(solve) i <<- solve
    getSolved <- function() i
    list(set = set, get = get,
    setSolved = setSolved,
    getSolved = getSolved)
    
}


## cacheSolve computes the inverse of the special "matrix" returned by
## makeCacheMatrix above. If the inverse has already been calculated (and the
## matrix has not changed), then the cachesolve should retrieve the inverse from
## the cache.

cacheSolve <- function(x, ...) {
    ## Return a matrix that is the inverse of 'x'
    i <- x$getSolved()
    if(!is.null(i)) {
        message("getting cached data")
        return(i)
    }
    data <- x$get()
    i <- solve(data, ...)
    x$setSolved(i)
    i  
}
\end{lstlisting}
%
%Листинг~\ref{list:external1} подгружается из внешнего файла. Приходится загружать без окружения дополнительного. Иначе по страницам не переносится.
%    \lstinputlisting[lastline=78,language={R},caption={Листинг из внешнего файла},label={list:external1}]{./listings/run_analysis.R}





\begin{comment}

\chapter{Очень длинное название второго приложения, в котором продемонстрирована работа с длинными таблицами} \label{AppendixB}

 \section{Подраздел приложения}\label{AppendixB1}
Вот размещается длинная таблица:
\fontsize{10pt}{10pt}\selectfont
\begin{longtable}[c]{|l|c|l|l|}
% \caption{Описание входных файлов модели}\label{Namelists} 
%\\ 
 \hline
 %\multicolumn{4}{|c|}{\textbf{Файл puma\_namelist}}        \\ \hline
 Параметр & Умолч. & Тип & Описание               \\ \hline
                                              \endfirsthead   \hline
 \multicolumn{4}{|c|}{\small\slshape (продолжение)}        \\ \hline
 Параметр & Умолч. & Тип & Описание               \\ \hline
                                              \endhead        \hline
 \multicolumn{4}{|r|}{\small\slshape продолжение следует}  \\ \hline
                                              \endfoot        \hline
                                              \endlastfoot
 \multicolumn{4}{|l|}{\&INP}        \\ \hline 
 kick & 1 & int & 0: инициализация без шума ($p_s = const$) \\
      &   &     & 1: генерация белого шума                  \\
      &   &     & 2: генерация белого шума симметрично относительно \\
  & & & экватора    \\
 mars & 0 & int & 1: инициализация модели для планеты Марс     \\
 kick & 1 & int & 0: инициализация без шума ($p_s = const$) \\
      &   &     & 1: генерация белого шума                  \\
      &   &     & 2: генерация белого шума симметрично относительно \\
  & & & экватора    \\
 mars & 0 & int & 1: инициализация модели для планеты Марс     \\
kick & 1 & int & 0: инициализация без шума ($p_s = const$) \\
      &   &     & 1: генерация белого шума                  \\
      &   &     & 2: генерация белого шума симметрично относительно \\
  & & & экватора    \\
 mars & 0 & int & 1: инициализация модели для планеты Марс     \\
kick & 1 & int & 0: инициализация без шума ($p_s = const$) \\
      &   &     & 1: генерация белого шума                  \\
      &   &     & 2: генерация белого шума симметрично относительно \\
  & & & экватора    \\
 mars & 0 & int & 1: инициализация модели для планеты Марс     \\
kick & 1 & int & 0: инициализация без шума ($p_s = const$) \\
      &   &     & 1: генерация белого шума                  \\
      &   &     & 2: генерация белого шума симметрично относительно \\
  & & & экватора    \\
 mars & 0 & int & 1: инициализация модели для планеты Марс     \\
kick & 1 & int & 0: инициализация без шума ($p_s = const$) \\
      &   &     & 1: генерация белого шума                  \\
      &   &     & 2: генерация белого шума симметрично относительно \\
  & & & экватора    \\
 mars & 0 & int & 1: инициализация модели для планеты Марс     \\
kick & 1 & int & 0: инициализация без шума ($p_s = const$) \\
      &   &     & 1: генерация белого шума                  \\
      &   &     & 2: генерация белого шума симметрично относительно \\
  & & & экватора    \\
 mars & 0 & int & 1: инициализация модели для планеты Марс     \\
kick & 1 & int & 0: инициализация без шума ($p_s = const$) \\
      &   &     & 1: генерация белого шума                  \\
      &   &     & 2: генерация белого шума симметрично относительно \\
  & & & экватора    \\
 mars & 0 & int & 1: инициализация модели для планеты Марс     \\
kick & 1 & int & 0: инициализация без шума ($p_s = const$) \\
      &   &     & 1: генерация белого шума                  \\
      &   &     & 2: генерация белого шума симметрично относительно \\
  & & & экватора    \\
 mars & 0 & int & 1: инициализация модели для планеты Марс     \\
kick & 1 & int & 0: инициализация без шума ($p_s = const$) \\
      &   &     & 1: генерация белого шума                  \\
      &   &     & 2: генерация белого шума симметрично относительно \\
  & & & экватора    \\
 mars & 0 & int & 1: инициализация модели для планеты Марс     \\
kick & 1 & int & 0: инициализация без шума ($p_s = const$) \\
      &   &     & 1: генерация белого шума                  \\
      &   &     & 2: генерация белого шума симметрично относительно \\
  & & & экватора    \\
 mars & 0 & int & 1: инициализация модели для планеты Марс     \\
kick & 1 & int & 0: инициализация без шума ($p_s = const$) \\
      &   &     & 1: генерация белого шума                  \\
      &   &     & 2: генерация белого шума симметрично относительно \\
  & & & экватора    \\
 mars & 0 & int & 1: инициализация модели для планеты Марс     \\
kick & 1 & int & 0: инициализация без шума ($p_s = const$) \\
      &   &     & 1: генерация белого шума                  \\
      &   &     & 2: генерация белого шума симметрично относительно \\
  & & & экватора    \\
 mars & 0 & int & 1: инициализация модели для планеты Марс     \\
kick & 1 & int & 0: инициализация без шума ($p_s = const$) \\
      &   &     & 1: генерация белого шума                  \\
      &   &     & 2: генерация белого шума симметрично относительно \\
  & & & экватора    \\
 mars & 0 & int & 1: инициализация модели для планеты Марс     \\
kick & 1 & int & 0: инициализация без шума ($p_s = const$) \\
      &   &     & 1: генерация белого шума                  \\
      &   &     & 2: генерация белого шума симметрично относительно \\
  & & & экватора    \\
 mars & 0 & int & 1: инициализация модели для планеты Марс     \\
 \hline
  %& & & $\:$ \\ 
 \multicolumn{4}{|l|}{\&SURFPAR}        \\ \hline
kick & 1 & int & 0: инициализация без шума ($p_s = const$) \\
      &   &     & 1: генерация белого шума                  \\
      &   &     & 2: генерация белого шума симметрично относительно \\
  & & & экватора    \\
 mars & 0 & int & 1: инициализация модели для планеты Марс     \\
kick & 1 & int & 0: инициализация без шума ($p_s = const$) \\
      &   &     & 1: генерация белого шума                  \\
      &   &     & 2: генерация белого шума симметрично относительно \\
  & & & экватора    \\
 mars & 0 & int & 1: инициализация модели для планеты Марс     \\
kick & 1 & int & 0: инициализация без шума ($p_s = const$) \\
      &   &     & 1: генерация белого шума                  \\
      &   &     & 2: генерация белого шума симметрично относительно \\
  & & & экватора    \\
 mars & 0 & int & 1: инициализация модели для планеты Марс     \\
kick & 1 & int & 0: инициализация без шума ($p_s = const$) \\
      &   &     & 1: генерация белого шума                  \\
      &   &     & 2: генерация белого шума симметрично относительно \\
  & & & экватора    \\
 mars & 0 & int & 1: инициализация модели для планеты Марс     \\
kick & 1 & int & 0: инициализация без шума ($p_s = const$) \\
      &   &     & 1: генерация белого шума                  \\
      &   &     & 2: генерация белого шума симметрично относительно \\
  & & & экватора    \\
 mars & 0 & int & 1: инициализация модели для планеты Марс     \\
kick & 1 & int & 0: инициализация без шума ($p_s = const$) \\
      &   &     & 1: генерация белого шума                  \\
      &   &     & 2: генерация белого шума симметрично относительно \\
  & & & экватора    \\
 mars & 0 & int & 1: инициализация модели для планеты Марс     \\
kick & 1 & int & 0: инициализация без шума ($p_s = const$) \\
      &   &     & 1: генерация белого шума                  \\
      &   &     & 2: генерация белого шума симметрично относительно \\
  & & & экватора    \\
 mars & 0 & int & 1: инициализация модели для планеты Марс     \\
kick & 1 & int & 0: инициализация без шума ($p_s = const$) \\
      &   &     & 1: генерация белого шума                  \\
      &   &     & 2: генерация белого шума симметрично относительно \\
  & & & экватора    \\
 mars & 0 & int & 1: инициализация модели для планеты Марс     \\
kick & 1 & int & 0: инициализация без шума ($p_s = const$) \\
      &   &     & 1: генерация белого шума                  \\
      &   &     & 2: генерация белого шума симметрично относительно \\
  & & & экватора    \\
 mars & 0 & int & 1: инициализация модели для планеты Марс     \\ 
 \hline 
\end{longtable}

\normalsize% возвращаем шрифт к нормальному
\end{comment}

\begin{comment}
\section{Ещё один подраздел приложения} \label{AppendixB2}

Нужно больше подразделов приложения!

Пример длинной таблицы с записью продолжения по ГОСТ 2.105

    \centering
	\small
    \begin{longtable}[c]{|l|c|l|l|}
	\caption{Наименование таблицы средней длины}%
    \label{tbl:test5}% label всегда желательно идти после caption
    \\ 
    \hline
     %\multicolumn{4}{|c|}{\textbf{Файл puma\_namelist}}        \\ \hline
     Параметр & Умолч. & Тип & Описание               \\ \hline
                                                  \endfirsthead
%     \multicolumn{4}{|c|}{\small\slshape (продолжение)}        \\ \hline
 \captionsetup{format=tablenocaption,labelformat=continued}% должен стоять до самого caption
 \caption[]{} \\
    \hline
     Параметр & Умолч. & Тип & Описание               \\ \hline
                                                  \endhead        \hline
%     \multicolumn{4}{|r|}{\small\slshape продолжение следует}  \\
%\hline
                                                  \endfoot        \hline
                                                  \endlastfoot
     \multicolumn{4}{|l|}{\&INP}        \\ \hline 
     kick & 1 & int & 0: инициализация без шума ($p_s = const$) \\
          &   &     & 1: генерация белого шума                  \\
          &   &     & 2: генерация белого шума симметрично относительно \\
      & & & экватора    \\
     mars & 0 & int & 1: инициализация модели для планеты Марс     \\
     kick & 1 & int & 0: инициализация без шума ($p_s = const$) \\
          &   &     & 1: генерация белого шума                  \\
          &   &     & 2: генерация белого шума симметрично относительно \\
      & & & экватора    \\
     mars & 0 & int & 1: инициализация модели для планеты Марс     \\
    kick & 1 & int & 0: инициализация без шума ($p_s = const$) \\
          &   &     & 1: генерация белого шума                  \\
          &   &     & 2: генерация белого шума симметрично относительно \\
      & & & экватора    \\
     mars & 0 & int & 1: инициализация модели для планеты Марс     \\
    kick & 1 & int & 0: инициализация без шума ($p_s = const$) \\
          &   &     & 1: генерация белого шума                  \\
          &   &     & 2: генерация белого шума симметрично относительно \\
      & & & экватора    \\
     mars & 0 & int & 1: инициализация модели для планеты Марс     \\
    kick & 1 & int & 0: инициализация без шума ($p_s = const$) \\
          &   &     & 1: генерация белого шума                  \\
          &   &     & 2: генерация белого шума симметрично относительно \\
      & & & экватора    \\
     mars & 0 & int & 1: инициализация модели для планеты Марс     \\
    kick & 1 & int & 0: инициализация без шума ($p_s = const$) \\
          &   &     & 1: генерация белого шума                  \\
          &   &     & 2: генерация белого шума симметрично относительно \\
      & & & экватора    \\
     mars & 0 & int & 1: инициализация модели для планеты Марс     \\
    kick & 1 & int & 0: инициализация без шума ($p_s = const$) \\
          &   &     & 1: генерация белого шума                  \\
          &   &     & 2: генерация белого шума симметрично относительно \\
      & & & экватора    \\
     mars & 0 & int & 1: инициализация модели для планеты Марс     \\
    kick & 1 & int & 0: инициализация без шума ($p_s = const$) \\
          &   &     & 1: генерация белого шума                  \\
          &   &     & 2: генерация белого шума симметрично относительно \\
      & & & экватора    \\
     mars & 0 & int & 1: инициализация модели для планеты Марс     \\
    kick & 1 & int & 0: инициализация без шума ($p_s = const$) \\
          &   &     & 1: генерация белого шума                  \\
          &   &     & 2: генерация белого шума симметрично относительно \\
      & & & экватора    \\
     mars & 0 & int & 1: инициализация модели для планеты Марс     \\
    kick & 1 & int & 0: инициализация без шума ($p_s = const$) \\
          &   &     & 1: генерация белого шума                  \\
          &   &     & 2: генерация белого шума симметрично относительно \\
      & & & экватора    \\
     mars & 0 & int & 1: инициализация модели для планеты Марс     \\
    kick & 1 & int & 0: инициализация без шума ($p_s = const$) \\
          &   &     & 1: генерация белого шума                  \\
          &   &     & 2: генерация белого шума симметрично относительно \\
      & & & экватора    \\
     mars & 0 & int & 1: инициализация модели для планеты Марс     \\
    kick & 1 & int & 0: инициализация без шума ($p_s = const$) \\
          &   &     & 1: генерация белого шума                  \\
          &   &     & 2: генерация белого шума симметрично относительно \\
      & & & экватора    \\
     mars & 0 & int & 1: инициализация модели для планеты Марс     \\
    kick & 1 & int & 0: инициализация без шума ($p_s = const$) \\
          &   &     & 1: генерация белого шума                  \\
          &   &     & 2: генерация белого шума симметрично относительно \\
      & & & экватора    \\
     mars & 0 & int & 1: инициализация модели для планеты Марс     \\
    kick & 1 & int & 0: инициализация без шума ($p_s = const$) \\
          &   &     & 1: генерация белого шума                  \\
          &   &     & 2: генерация белого шума симметрично относительно \\
      & & & экватора    \\
     mars & 0 & int & 1: инициализация модели для планеты Марс     \\
    kick & 1 & int & 0: инициализация без шума ($p_s = const$) \\
          &   &     & 1: генерация белого шума                  \\
          &   &     & 2: генерация белого шума симметрично относительно \\
      & & & экватора    \\
     mars & 0 & int & 1: инициализация модели для планеты Марс     \\
     \hline
      %& & & $\:$ \\ 
     \multicolumn{4}{|l|}{\&SURFPAR}        \\ \hline
    kick & 1 & int & 0: инициализация без шума ($p_s = const$) \\
          &   &     & 1: генерация белого шума                  \\
          &   &     & 2: генерация белого шума симметрично относительно \\
      & & & экватора    \\
     mars & 0 & int & 1: инициализация модели для планеты Марс     \\
    kick & 1 & int & 0: инициализация без шума ($p_s = const$) \\
          &   &     & 1: генерация белого шума                  \\
          &   &     & 2: генерация белого шума симметрично относительно \\
      & & & экватора    \\
     mars & 0 & int & 1: инициализация модели для планеты Марс     \\
    kick & 1 & int & 0: инициализация без шума ($p_s = const$) \\
          &   &     & 1: генерация белого шума                  \\
          &   &     & 2: генерация белого шума симметрично относительно \\
      & & & экватора    \\
     mars & 0 & int & 1: инициализация модели для планеты Марс     \\
    kick & 1 & int & 0: инициализация без шума ($p_s = const$) \\
          &   &     & 1: генерация белого шума                  \\
          &   &     & 2: генерация белого шума симметрично относительно \\
      & & & экватора    \\
     mars & 0 & int & 1: инициализация модели для планеты Марс     \\
    kick & 1 & int & 0: инициализация без шума ($p_s = const$) \\
          &   &     & 1: генерация белого шума                  \\
          &   &     & 2: генерация белого шума симметрично относительно \\
      & & & экватора    \\
     mars & 0 & int & 1: инициализация модели для планеты Марс     \\
    kick & 1 & int & 0: инициализация без шума ($p_s = const$) \\
          &   &     & 1: генерация белого шума                  \\
          &   &     & 2: генерация белого шума симметрично относительно \\
      & & & экватора    \\
     mars & 0 & int & 1: инициализация модели для планеты Марс     \\
    kick & 1 & int & 0: инициализация без шума ($p_s = const$) \\
          &   &     & 1: генерация белого шума                  \\
          &   &     & 2: генерация белого шума симметрично относительно \\
      & & & экватора    \\
     mars & 0 & int & 1: инициализация модели для планеты Марс     \\
    kick & 1 & int & 0: инициализация без шума ($p_s = const$) \\
          &   &     & 1: генерация белого шума                  \\
          &   &     & 2: генерация белого шума симметрично относительно \\
      & & & экватора    \\
     mars & 0 & int & 1: инициализация модели для планеты Марс     \\
    kick & 1 & int & 0: инициализация без шума ($p_s = const$) \\
          &   &     & 1: генерация белого шума                  \\
          &   &     & 2: генерация белого шума симметрично относительно \\
      & & & экватора    \\
     mars & 0 & int & 1: инициализация модели для планеты Марс     \\ 
     \hline 
    \end{longtable}
\normalsize% возвращаем шрифт к нормальному
\end{comment}