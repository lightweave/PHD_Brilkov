\chapter{Аппаратура для проведения исследований} \label{chapt2}

\section{Прибор Дэпрон}

Прибор Дэпрон разрабатывался как исследовательский инструмент для решения широкого круга научных задач. Основной задачей прибора является измерение мощности дозы и потоков ионизирующих излучений. Дополнительными задачами выделены регистрация нейтронов тепловых энергий и высокоэнергетичных частиц. Такое сочетание решаемых задач, для прибора относительно небольшого веса,  является уникальным и позволяет надеяться на получение достаточно подробной информации о радиационной обстановке на борту КА. 

\subsection{Устройство прибора}

В состав прибора ДЭПРОН входят два узла с полупроводниковыми детекторами и два узла с газоразрядными гелиевыми счетчиками нейтронов. Также в состав прибора входят узлы усиления и формирования сигналов от полупроводниковых и нейтронных детекторов и узел цифровой обработки сигналов.


Поглощенная доза регистрируется узлами с полупроводниковыми детекторами. Для получения информации о величине поглощенной дозы используется принцип регистрации величины заряда в объеме полупроводника, пропорционального энерговыделению в данном объеме. 

\todo{формула энерговыделения}

Оба полупроводниковых детектора и скомпонованы в кассету и расположены в относительной близости друг от друга. Схема построения прибора с параллельным расположением двух полупроводниковых детекторов была использована для получения информации о ЛПЭ частиц, прошедших одновременно оба детектора. 

\todo{Рисунок Блок-схема прибора ДЭПРОН}

%\newpage
%============================================================================================================================
\section{Конструктивные особенности прибора}

Прибор состоит из одного блока, габаритный чертеж которого представлен в Приложении 1. Габаритные размеры прибора: длина  280 мм, ширина 160 мм, высота 78 мм. Масса прибора - 3 кг. Корпус прибора составлен из шести пластин Д16т -- листового дюралюминия, толщиной 4,5 мм, обработанного на станке ЧПУ. В каждой пластине фрезерованы повторяющиеся выборки треугольной формы до толщины 2 мм. Выборки расположены таким образом, чтобы сформировать «ребра» жесткости в стенках прибора. С лицевой стороны пластины корпуса оксидированы, с целью получения электропроводной поверхности всего прибора.

\todo{Рисунок Вариант размещения выборок в днище прибора ДЭПРОН. В последствии данный вариант переработан и заменен исходя из конструктивных соображений крепления модулей электроники и улучшения теплосброса источников питания, через термоконтакт с бортом КА.}

На лицевом торце прибора распложены два разъема СНП-333, используемых для передачи данных в БИ аппаратуры спутника (разъем Х1) и для передачи питания в прибор ДЭПРОН от бортовой аппаратуры спутника (разъем Х2). Также на лицевой панели находятся два разъема РС-7 предназначенные для передачи информации по каналу РС232 от прибора ИМИСС-1 (разъем Х5) и сквозной передачи питания от бортовой аппаратуры к прибору ИМИСС-1 (разъем Х4). Во всех перечисленных разъемах предусмотрен контроль стыковки разъемов с помощью короткозамкнутых линий, а также дублирование информационных и токонесущих линий.


Дополнительно на лицевую панель прибора вынесен технологический разъем РС 19 ХТ3, используемый для проверки функционирования прибора в лабораторных условиях методом подачи на детекторные узлы калиброванных сигналов с генератора, а также для контроля внутренних рабочих напряжений. Проверка работоспособности прибора и подача сигналов с генератора осуществляется с помощью  блока КПА, имеющему четыре экранированных канала для передачи низкоамплитудных сигналов и два светодиодных индикатора для контроля наличия рабочих напряжений  $ +5  $В и$  +12 $ В в приборе ДЭПРОН. В штатном режиме работы данный разъем не подключен и закрыт заглушкой. Схема распределения линий в разъемах представлена в Приложении 2.

Платы электроники блоков усиления и формирования аналоговых сигналов располагаются в трех тонкостенных алюминиевых кассетах и выполнены в формате 11-ти контактных печатных плат размерами 34х50мм. Данный формат печатных плат распространен в производстве научной аппаратуры изготовления НИИЯФ МГУ и с успехом применяется для космической аппаратуры уже на протяжении нескольких десятков лет. Применение данного стандарта позволяет соблюсти принцип модульности построения приборов, используя отработанные в космических условиях надежные схемы, компонуя из них тракты с параметрами, заданными потребностями текущих экспериментальных задач. 

\todo{Рисунок Внутренняя компоновка модулей прибора ДЭПРОН. Вид сверху со снятой крышкой прибора. }

В средней части рисунка последовательно располагаются три корпусных кассеты с платами электроники: левая и правая кассеты содержат платы формирователей триггерных сигналов от детекторов, центральная кассета ориентирована перпендикулярно и содержит две платы полупроводниковых детекторов и ЗЧУ, а также платы дополнительного усиления.

В нижней части рисунка находится нейтронный счетчик СИ13Н (циллиндр), экранированный 1 см оргстекла

\section{Детекторы}

Дозиметр заряженных частиц выполнен на кремниевых ионно-имплантированных Д1 пролетных детекторах, работающих в режиме регистрации амплитуд импульсов. Детекторы изготовлены по специальному заказу НИИЯФ МГУ в ООО «Детектор-СИ» в соответствии с АБЛК.418219.402ТУ. Детекторы, в соответствии с паспортом, предназначены для спектрометрии и радиометрии заряженных частиц в составе предназначенной для этих целей аппаратуры. Детекторы могут эксплуатироваться при атмосферном давлении или в вакууме до 10\textsuperscript{-6} мм.рт.ст. чувствительный элемент детектора изготовлен из высокоомного кремния n--типа по технологии ионной имплантации.

Значения параметров детекторов приведены в таблице.

\todo{Наименование параметраФактические параметрыПримечанияРабочее напряжение, В90Аттестация производилась при 26 CОбратный ток, нА4Энергетический эквивалент шума, кэВ5Постоянная времени квазигауссова формирования импульса, мкс2Предельно допустимое напряжение, В130}

Рекомендуемая схема включения детектора приведена на рис. 


\todo{\\Рисунок Схема включения детектора.}


+Еп -- источник напряжения;


Rсм -- сопротивление смещения;


D1 -- Детектор;


Для обеспечения надежности используются два полупроводниковых детектора. Детекторы образуют телескоп, что обеспечивает возможность регистрировать спектр ионизационных потерь.

Детектор нейтронов выполнен на счётчике медленных нейтронов «СИ-13Н», представляющем собой газоразрядный  счетчик, работающий в режиме коронного разряда. Для обеспечения надежности используются 2 счетчика. Второй детектор нейтронов окружен замедляющей оболочной из поликарбоната, что позволило расширить энергетический диапазон регистрируемых нейтронов. При прохождении нейтрона через газ Не-3, наполняющий счетчик, происходит ядерная реакция:
\[ n+^3\!He = p+T+764 \textrm{ КэВ}\]
 Продукты реакции вызывают ионизацию газа в счётчике, что приводит к образованию газового разряда и появлению электрического импульса на электроде счетчика. Импульс поступает на вход усилителя-формирователя и, затем, поступает на регистр прерываний процессора, где используется для подсчета числа зарегистрированных нейтронов.

