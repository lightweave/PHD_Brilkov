\chapter{Обзор Литературы} \label{chapt1}

\section{Радиационная обстановка на высокоширотных околоземных орбитах. Вопросы, требующие детального исследования.} \label{sect1_1}
Исследования радиационной обстановки в космическом пространстве связано с началом полетов автоматических аппаратов и человека в космос.  Широкое распространение технологий, связанных с использованием космической техники, а также непрерывные пребывание человека в космическом пространстве во время миссий на космических станциях МИР и МКС позволило выявить ряд опасностей космических полетов, среди которых особое внимание следует уделить радиационной опасности \cite{logachev2007}.


Запуск 2-го и 3-го спутников Земли, с приборами, изготовленными в НИИЯФ МГУ,  показал принципиальную возможность полета человека в космос,  однако, как можно заметить из данных полученных при начальных исследованиях радиационной обстановки, на орбите земли существуют отдельные области повышения радиационного фона (Рисунок …). Существование данных областей связано с неоднородностями магнитного поля Земли и приводит к формированию области повышения потоков частиц в Южно Атлантической области \cite{logachev2007}, названной Южно-Атлантической Аномалией (ЮАА), как показано в статье Вернова С.Н.\cite{vernov1961} . В первом приближении для описания магнитного поля  Земли на высотах до 2000 км можно использовать представление модели смещенного диполя, этот подход позволяет учитывать ЮАА [Модель космоса 3 том 20стр].

%\newpage
%============================================================================================================================

\section{Ссылки} \label{sect1_2}
%Сошлёмся на библиографию. Одна ссылка: \cite[с.~54]{Sokolov}\cite[с.~36]{Gaidaenko}. Две ссылки: \cite{Sokolov,Gaidaenko}. Много ссылок:  \cite[с.~54]{Lermontov,Management,Borozda} \cite{Lermontov,Management,Borozda,Marketing,Constitution,FamilyCode,Gost.7.0.53,Razumovski,Lagkueva,Pokrovski,Sirotko,Lukina,Methodology,Encyclopedia,Nasirova,Berestova,Kriger}. И ещё немного ссылок: \cite{Article,Book,Booklet,Conference,Inbook,Incollection,Manual,Mastersthesis,Misc,Phdthesis,Proceedings,Techreport,Unpublished}. \cite{medvedev2006jelektronnye, CEAT:CEAT581, doi:10.1080/01932691.2010.513279,Gosele1999161,Li2007StressAnalysis, Shoji199895,test:eisner-sample,AB_patent_Pomerantz_1968,iofis_patent1960}

%Попытка реализовать несколько ссылок на конкретные страницы для стандартной реализации:[\citenum{Sokolov}, с.~54; \citenum{Gaidaenko}, с.~36].

%Несколько источников мультицитата \cites[vii--x, 5, 7]{Sokolov}[v--x, 25, 526]{Gaidaenko} поехали дальше


Сошлёмся на приложения: Приложение \ref{AppendixA}, Приложение \ref{AppendixB2}.

Сошлёмся на формулу: формула \eqref{eq:equation1}.

Сошлёмся на изображение: рисунок \ref{img:knuth}.

%\newpage
%============================================================================================================================

\section{Формулы} \label{sect1_3}

Благодаря пакету \textit{icomma}, \LaTeX~одинаково хорошо воспринимает в качестве десятичного разделителя и запятую ($3,1415$), и точку ($3.1415$).

\subsection{Ненумерованные одиночные формулы} \label{subsect1_3_1}

Вот так может выглядеть формула, которую необходимо вставить в строку по тексту: $x \approx \sin x$ при $x \to 0$.

А вот так выглядит ненумерованая отдельностоящая формула c подстрочными и надстрочными индексами:
\[
(x_1+x_2)^2 = x_1^2 + 2 x_1 x_2 + x_2^2
\]

При использовании дробей формулы могут получаться очень высокие:
\[
  \frac{1}{\sqrt(2)+
  \displaystyle\frac{1}{\sqrt{2}+
  \displaystyle\frac{1}{\sqrt{2}+\cdots}}}
\]

В формулах можно использовать греческие буквы:
\[
\alpha\beta\gamma\delta\epsilon\varepsilon\zeta\eta\theta\vartheta\iota\kappa\lambda\\mu\nu\xi\pi\varpi\rho\varrho\sigma\varsigma\tau\upsilon\phi\varphi\chi\psi\omega\Gamma\Delta\Theta\Lambda\Xi\Pi\Sigma\Upsilon\Phi\Psi\Omega
\]

%\newpage
%============================================================================================================================

\subsection{Ненумерованные многострочные формулы} \label{subsect1_3_2}

Вот так можно написать две формулы, не нумеруя их, чтобы знаки равно были строго друг под другом:
\begin{align}
  f_W & =  \min \left( 1, \max \left( 0, \frac{W_{soil} / W_{max}}{W_{crit}} \right)  \right), \nonumber \\
  f_T & =  \min \left( 1, \max \left( 0, \frac{T_s / T_{melt}}{T_{crit}} \right)  \right), \nonumber
\end{align}

Выровнять систему ещё и по переменной $ x $ можно, используя окружение \verb|alignedat| из пакета \verb|amsmath|. Вот так: 
\[
    |x| = \left\{
    \begin{alignedat}{2}
        &&x, \quad &\text{eсли } x\geqslant 0 \\
        &-&x, \quad & \text{eсли } x<0
    \end{alignedat}
    \right.
\]
Здесь первый амперсанд  означает выравнивание по~левому краю, второй "--- по~$ x $, а~третий "--- по~слову <<если>>. Команда \verb|\quad| делает большой горизонтальный пробел. 

Ещё вариант:
\[
    |x|=
    \begin{cases}
    \phantom{-}x, \text{если } x \geqslant 0 \\
    -x, \text{если } x<0
    \end{cases}
\]

Можно использовать разные математические алфавиты:
\begin{align}
\mathcal{ABCDEFGHIJKLMNOPQRSTUVWXYZ} \nonumber \\
\mathfrak{ABCDEFGHIJKLMNOPQRSTUVWXYZ} \nonumber \\
\mathbb{ABCDEFGHIJKLMNOPQRSTUVWXYZ} \nonumber
\end{align}

Посмотрим на систему уравнений на примере аттрактора Лоренца:

\[ 
\left\{
  \begin{array}{rl}
    \dot x = & \sigma (y-x) \\
    \dot y = & x (r - z) - y \\
    \dot z = & xy - bz
  \end{array}
\right.
\]

А для вёрстки матриц удобно использовать многоточия:
\[ 
\left(
  \begin{array}{ccc}
  	a_{11} & \ldots & a_{1n} \\
  	\vdots & \ddots & \vdots \\
  	a_{n1} & \ldots & a_{nn} \\
  \end{array}
\right)
\]


%\newpage
%============================================================================================================================
\subsection{Нумерованные формулы} \label{subsect1_3_3}

А вот так пишется нумерованая формула:
\begin{equation}
  \label{eq:equation1}
  e = \lim_{n \to \infty} \left( 1+\frac{1}{n} \right) ^n
\end{equation}

Нумерованых формул может быть несколько:
\begin{equation}
  \label{eq:equation2}
  \lim_{n \to \infty} \sum_{k=1}^n \frac{1}{k^2} = \frac{\pi^2}{6}
\end{equation}

Впоследствии на формулы (\ref{eq:equation1}) и (\ref{eq:equation2}) можно ссылаться.

Сделать так, чтобы номер формулы стоял напротив средней строки, можно, используя окружение \verb|multlined| (пакет \verb|mathtools|) вместо \verb|multline| внутри окружения \verb|equation|. Вот так:
\begin{equation} % \tag{S} % tag - вписывает свой текст 
    \begin{multlined}
        1+ 2+3+4+5+6+7+\dots + \\ 
        + 50+51+52+53+54+55+56+57 + \dots + \\ 
        + 96+97+98+99+100=5050 
    \end{multlined}
\end{equation}
