% Формат А4, 14pt (ГОСТ Р 7.0.11-2011, 5.3.6)
\documentclass[a4paper,14pt]{extreport}

\input{setup}               % Упрощённые настройки шаблона 

%%% Проверка используемого TeX-движка %%%
\usepackage{iftex}
\newif\ifxetexorluatex   % определяем новый условный оператор (http://tex.stackexchange.com/a/47579/79756)
\ifXeTeX
    \xetexorluatextrue
\else
    \ifLuaTeX
        \xetexorluatextrue
    \else
        \xetexorluatexfalse
    \fi
\fi

%%% Поля и разметка страницы %%%
\usepackage{pdflscape}                              % Для включения альбомных страниц
\usepackage{geometry}                               % Для последующего задания полей

%%% Математические пакеты %%%
\usepackage{amsthm,amsfonts,amsmath,amssymb,amscd}  % Математические дополнения от AMS
\usepackage{mathtools}                              % Добавляет окружение multlined

%%%% Установки для размера шрифта 14 pt %%%%
%% Формирование переменных и констант для сравнения (один раз для всех подключаемых файлов)%%
%% должно располагаться до вызова пакета fontspec или polyglossia, потому что они сбивают его работу
\newlength{\curtextsize}
\newlength{\bigtextsize}
\setlength{\bigtextsize}{13.9pt}

\makeatletter
%\show\f@size                                       % неплохо для отслеживания, но вызывает стопорение процесса, если документ компилируется без команды  -interaction=nonstopmode 
\setlength{\curtextsize}{\f@size pt}
\makeatother

%%% Кодировки и шрифты %%%
\ifxetexorluatex
	
    \usepackage{polyglossia}                        % Поддержка многоязычности (fontspec подгружается автоматически)
    \usepackage{csquotes}
\else
    \RequirePDFTeX                                  % tests for PDFTEX use and throws an error if a different engine is being used
    \usepackage{cmap}                               % Улучшенный поиск русских слов в полученном pdf-файле
    \usepackage[T2A]{fontenc}                       % Поддержка русских букв
    \usepackage[utf8]{inputenc}                     % Кодировка utf8
    \usepackage[english, russian]{babel}            % Языки: русский, английский
    \usepackage{csquotes}
    \IfFileExists{pscyr.sty}{\usepackage{pscyr}}{}  % Красивые русские шрифты
\fi

%%% Оформление абзацев %%%
\usepackage{indentfirst}                            % Красная строка

%%% Цвета %%%
\usepackage[dvipsnames,usenames]{color}
\usepackage{colortbl}

%%% Таблицы %%%
\usepackage{longtable}                              % Длинные таблицы
\usepackage{multirow,makecell,array}                % Улучшенное форматирование таблиц
\usepackage{booktabs}                               % Возможность оформления таблиц в классическом книжном стиле (при правильном использовании не противоречит ГОСТ)

%%% Общее форматирование
\usepackage{soulutf8}                               % Поддержка переносоустойчивых подчёркиваний и зачёркиваний
\usepackage{icomma}                                 % Запятая в десятичных дробях


%%% Гиперссылки %%%
\usepackage{hyperref}

%%% Изображения %%%
\usepackage{graphicx}                               % Подключаем пакет работы с графикой

%%% Графика %%%
\usepackage{tikz}
\usetikzlibrary{calc}
\usetikzlibrary{quotes,arrows.meta}

\tikzset{
	annotated cuboid/.pic={
		\tikzset{%
			every edge quotes/.append style={midway, auto},
			/cuboid/.cd,
			#1
		}
		\draw [every edge/.append style={pic actions, densely dashed, opacity=.5}, pic actions]
		(0,0,0) coordinate (o) -- ++(-\cubescale*\cubex,0,0) coordinate (a) -- ++(0,-\cubescale*\cubey,0) coordinate (b) edge coordinate [pos=1] (g) ++(0,0,-\cubescale*\cubez)  -- ++(\cubescale*\cubex,0,0) coordinate (c) -- cycle
		(o) -- ++(0,0,-\cubescale*\cubez) coordinate (d) -- ++(0,-\cubescale*\cubey,0) coordinate (e) edge (g) -- (c) -- cycle
		(o) -- (a) -- ++(0,0,-\cubescale*\cubez) coordinate (f) edge (g) -- (d) -- cycle;
		\path [every edge/.append style={pic actions, |-|}]
		(b) +(0,-5pt) coordinate (b1) edge ["\cubex \cubeunits"'] (b1 -| c)
		(b) +(-5pt,0) coordinate (b2) edge ["\cubey \cubeunits"] (b2 |- a)
		(c) +(3.5pt,-3.5pt) coordinate (c2) edge ["\cubez \cubeunits"'] ([xshift=3.5pt,yshift=-3.5pt]e)
		;
	},
	/cuboid/.search also={/tikz},
	/cuboid/.cd,
	width/.store in=\cubex,
	height/.store in=\cubey,
	depth/.store in=\cubez,
	units/.store in=\cubeunits,
	scale/.store in=\cubescale,
	width=10,
	height=10,
	depth=10,
	units=cm,
	scale=.1,
}

%%% Списки %%%
\usepackage{enumitem}

%%% Подписи %%%
\usepackage{caption}                                % Для управления подписями (рисунков и таблиц) % Может управлять номерами рисунков и таблиц с caption %Иногда может управлять заголовками в списках рисунков и таблиц
\usepackage{subcaption}                             % Работа с подрисунками и подобным

%%% Интервалы %%%
\usepackage[onehalfspacing]{setspace}               % Опция запуска пакета правит не только интервалы в обычном тексте, но и формульные

\usepackage{layout}
 \usepackage{rotating}

%%% Счётчики %%%
\usepackage[figure,table]{totalcount}               % Счётчик рисунков и таблиц
\usepackage{totcount}                               % Пакет создания счётчиков на основе последнего номера подсчитываемого элемента (может требовать дважды компилировать документ)
\usepackage{totpages}                               % Счётчик страниц, совместимый с hyperref (ссылается на номер последней страницы). Желательно ставить последним пакетом в преамбуле
%%% Многострочные комментарии %%%
\usepackage{comment}  % Пакеты общие для диссертации и автореферата
\input{dispackages}         % Пакеты для диссертации
\usepackage{tabularx,tabulary}  %таблицы с автоматически подбирающейся шириной столбцов

% Листинги с исходным кодом программ
\usepackage{fancyvrb}
\usepackage{listings}
%\newcommand\TestAppExists[3]{#2}
%\usepackage{minted}

% Плавающие окружения. во многом лучше пакета float
\usepackage{floatrow}

\usepackage{stackengine}

%\usepackage{rotating}

%\usepackage{showframe}        % Пакеты для специфических пользовательских задач
\input{preamblenames}       % Переопределение именований, чтобы можно было и в преамбуле использовать
\input{../common/styles}    % Стили общие для диссертации и автореферата
\input{disstyles}           % Стили для диссертации
\input{userstyles}          % Стили для специфических пользовательских задач
\input{../biblio/bibliopreamble}% Настройки библиографии из внешнего файла (там же выбор: встроенная или на основе biblatex)
%%% Основные сведения %%%
\newcommand{\thesisAuthor}             % Диссертация, ФИО автора
{Золотарев Иван Анатольевич}
\newcommand{\thesisUdk}                % Диссертация, УДК
{\523.4-854}
\newcommand{\thesisTitle}              % Диссертация, название
{\MakeUppercase{ОПРЕДЕЛЕНИЕ РАДИАЦИОННОЙ НАГРУЗКИ В КОСМИЧЕСКОМ АППАРАТЕ ПРИ ПОЛЕТЕ ПО ВЫСОКОШИРОТНОЙ ОРБИТЕ}}
\newcommand{\thesisSpecialtyNumber}    % Диссертация, специальность, номер
{05.26.02}
\newcommand{\thesisSpecialtyTitle}     % Диссертация, специальность, название
{{Безопасность в чрезвычайных ситуациях\\ (авиационная и ракетно-космическая техника)}}
\newcommand{\thesisDegree}             % Диссертация, научная степень
{{кандидата технических наук}}
\newcommand{\thesisCity}               % Диссертация, город защиты
{{Москва}}
\newcommand{\thesisYear}               % Диссертация, год защиты
{{2017}}
\newcommand{\thesisOrganization}       % Диссертация, организация
{{Научно-исследовательский институт ядерной физики имени Д.В.Скобельцына}}

\newcommand{\supervisorFio}            % Научный руководитель, ФИО
{{Бенгин Виктор Владимирович}}
\newcommand{\supervisorRegalia}        % Научный руководитель, регалии
{{кандидат физико-математических наук}}

\newcommand{\opponentOneFio}           % Оппонент 1, ФИО
{\todo{Фамилия Имя Отчество}}
\newcommand{\opponentOneRegalia}       % Оппонент 1, регалии
{\todo{доктор физико-математических наук, профессор}}
\newcommand{\opponentOneJobPlace}      % Оппонент 1, место работы
{\todo{Не очень длинное название для места работы}}
\newcommand{\opponentOneJobPost}       % Оппонент 1, должность
{\todo{старший научный сотрудник}}

\newcommand{\opponentTwoFio}           % Оппонент 2, ФИО
{\todo{Фамилия Имя Отчество}}
\newcommand{\opponentTwoRegalia}       % Оппонент 2, регалии
{\todo{кандидат физико-математических наук}}
\newcommand{\opponentTwoJobPlace}      % Оппонент 2, место работы
{\todo{Основное место работы c длинным длинным длинным длинным названием}}
\newcommand{\opponentTwoJobPost}       % Оппонент 2, должность
{\todo{старший научный сотрудник}}

\newcommand{\leadingOrganizationTitle} % Ведущая организация, дополнительные строки
{Научно-исследовательский институт ядерной физики имени Д.В.Скобельцына}

\newcommand{\defenseDate}              % Защита, дата
{\todo{DD mmmmmmmm YYYY~г.~в~XX часов}}
\newcommand{\defenseCouncilNumber}     % Защита, номер диссертационного совета
{\todo{NN}}
\newcommand{\defenseCouncilTitle}      % Защита, учреждение диссертационного совета
{\todo{Название учреждения}}
\newcommand{\defenseCouncilAddress}    % Защита, адрес учреждение диссертационного совета
{\todo{Адрес}}

\newcommand{\defenseSecretaryFio}      % Секретарь диссертационного совета, ФИО
{\todo{Фамилия Имя Отчество}}
\newcommand{\defenseSecretaryRegalia}  % Секретарь диссертационного совета, регалии
{\todo{д-р~физ.-мат. наук}}            % Для сокращений есть ГОСТы, например: ГОСТ Р 7.0.12-2011 + http://base.garant.ru/179724/#block_30000

\newcommand{\synopsisLibrary}          % Автореферат, название библиотеки
{\todo{Название библиотеки}}
\newcommand{\synopsisDate}             % Автореферат, дата рассылки
{\todo{DD mmmmmmmm YYYY года}}
      % Основные сведения

\begin{document}
	\chapter{Восстановление метки времени в массивах данных}

	В процессе обработки данных полученных во время летных испытаний прибора ДЭПРОН 
	было выяснено, что во временных метках массивов информации имеются ошибочные 
	значения, происхождение которых связано с отсутствием календаря в ПО 
	микроконтроллера ДЭПРОН. Так как в ПО не были заложены длительности месяцов 
	года, при наступлении нового месяца метки времени продолжают приходить с 
	номером предыдущего месяца к числу дней прибавляется дополнительный и возникают 
	ошибочные даты: 2016-05-32 и 2016-05-33. На рисунке 
	\ref{fig:deprontimedifference} видно наличие пробелов при наступлении нового 
	месяца, так как невозможно автоматическое распознание меток времени. Наличие 
	таких отклонений должно быть 
	исправлено отправлением метки времени в прибор от БИ в первую минуту нового 
	месяца. Однако пока такая процедура не проводилась был накоплен значительный 
	объем измерений и для их верной привязки к действительным датам был разработан 
	алгоритм и реализован на языке R, листинг кода представлен в \ref{list:datecor}.
	
	\begin{figure}
		\centering
		\includegraphics[width=0.8\linewidth]{images/deprontimedifference}
		\caption[Временной ряд разницы приборного времени и меток начала записи в 
		файл.]{Временной ряд разницы приборного времени и меток начала записи в 
			файл. Показаны первые шесть месяцев после запуска спутника Ломоносов, 
			отключения прибора соответствуют циклограмме летных испытаний, а 
			пробелы в данных в начале месяца соответствуют ошибочным номерам дня в 
			месяце.}
		\label{fig:deprontimedifference}
	\end{figure}
	
	При привязке секундных данных к баллистическим данным была обнаружена еще одна 
	проблема с метками времени в данных 
	ДЭПРОН, связанная с постоянным уходом приборных часов. Для решения этой 
	проблемы также был разработан алгоритм \ref{list:timecor} и успешно применен 
	для восстановления меток времени 
	
	\subsubsection{Описание алгоритма восстановления дат}
	
	На первом этапе  бинарные данные каждого сброса распаковываются в текстовый вид 
	с получением таблицы с колонками: \texttt{YYYY-MM-DD hh:mm:ss-1s,	count.h,	
		count1,	count2,	count.both,	n1,	n2,	dose1, dose2,	filename,	timestamp}. 
	Далее осуществляется разделение текстового поля с меткой времени --- 
	\texttt{YYYY-MM-DD hh:mm:ss-1s} на дату и время, а после полученное поле 
	\texttt{date} на год, месяц и день - обозначенные \texttt{year, month, day} 
	соотвественно.
	
	Создается поле \texttt{dates} имеющее тип данных \texttt{ISOdate} исходя только 
	из полей даты года и месяца, а день месяца устанавливается первый. Далее к полю 
	\texttt{dates} 
	добавляется число дней из поля \texttt{day}, минус один день. В последнюю 
	очередь в поле \texttt{dates} выставляется приборное время, отделенное в начале 
	алгоритма.
	
	\subsubsection{Описание алгоритма восстановления метки времени}
	
	Приборное время ДЭПРОН установлено на третий часовой пояс и соотвествует 
	Московскому 
	времени, поэтому для унификации базы данных получено поле \texttt{dates.UTC} 
	соответствующее приборному времени смещенному на 3 часа.
	
	
	
	Далее в результате ручного анализа данных было найдено что за сутки внутренние 
	часы ДЭПРОН уходят вперед на 57 секунд, что хорошо видно на графике~
	\ref{fig:deprontimedifference} поэтому был введена поправка \texttt{kt}:
	\[ kt = (56.77315002) /86400 \]
	Далее c использованием полученной поправки из времени UTC получено 
	скорректированное приборное время, хотя это время смещено отностительно 
	действительного всемирного времени это смещение меняется каждый раз при 
	выключениях прибора в ходе программы исследований \ref{fig:deprontimeplot}. 
	
	\begin{figure}
		\centering
		\includegraphics[width=0.8\linewidth]{images/deprontimeplot}
		\caption[Временной ряд разницы калиброванного приборного времени и меток 
		начала записи в файл.]{Временной ряд разницы калиброванного приборного 
			времени и меток начала записи в файл.}
		\label{fig:deprontimeplot}
	\end{figure}
	
	
	Для автоматического определения этого смещения необходимо привязать его к 
	независимому источнику точного времени - в качестве такого рассмотрены метки 
	времени начала записи в бинарный файл данных БИ а также метки окончания записи 
	файловой системы. Оказалось что разница времени меток последней записи сильно 
	разнится относительно всех других меток, видимо по причине буферизации записи в 
	файл данных, поэтому в качестве реперного времени выбрано время создания файла, 
	которое записано в названии каждого файла как POSIXtime в виде 
	шестнадцатеричного числа.
	
	После получения разницы ``горизонтального'' приборного времени с временем 
	начала записи в файл (поле time.delta.file.start) мы рассчитываем минимум 
	(\texttt{delta.minimum}) этой разницы для каждого файла бинарных данных. Анализ 
	распределения минимумов показал что моменты ``перескоков'' приборного времени 
	из-за выключений приводят к разницам более двух минут, таким образом времена 
	выключений были отсеяны от минутных или двухминутных пропусков в при записи в 
	бинарных файлах данных. На основе данных о перескоках времени составлен массив 
	\texttt{data.sec.switches}, который записывается в отдельный файл и также 
	исходный массив секундных данных разбивается на участки без выключений прибора. 
	
	Для каждого участка непрерывной работы были найдены наиболее часто 
	встречающиеся значения смещений \texttt{mfv.delta} --- мода разниц. И в 
	соответствии с этими значениями скорректировано приборное время.
	
	Последней операцией производится смещение полученного правильно времени на 59 
	секунд назад, так как 
	
	
	
	
	
	
	\begin{figure}
		\centering
		\includegraphics[width=0.9\linewidth]{images/depron_time_172}
		\caption{Пример восстановления меток времени 20 июня 2016 года}
		\label{fig:deprontime172}
	\end{figure}
	
	
	
	
	\section{Листинги программного кода комплекса ДЭПРОН} \label{AppendixB}
	
	
	По причине проблем с поддержкой кириллицы (она встречается в комментариях и
	печатаемых сообщениях), комментарии не отображены ~\ref{list:datecor}.
	%\renewcommand\FBbskip{-20pt} % если хотим притянуть что-то к плавающему окружению из floatrow
	
	\begin{ListingEnv}[H]
		% элементы, которые нежелательно разрывать обычно не ставят
		% посреди страницы: вместо H используется t (top, сверху страницы),
		% или b (bottom) или p (page, на отдельной странице)
		%    \captionsetup{format=tablenocaption}% должен стоять до самого caption
		%    \thisfloatsetup{\capposition=top}%
		\caption{Алгоритм коррекции даты в начале нового месяца на языке R}
		% далее метка для ссылки:
		\label{list:datecor}
		\begin{lstlisting}[language={Renhanced}]
		# date correction---------------------------------------------------------
		
		data.sec<-separate(data.sec, 'YYYY-MM-DD hh:mm:ss-1s',c("date", "time"),
		sep=' ')
		data.sec<-separate(data.sec, 'date',c("year", "month", "day"),
		sep='-', convert = TRUE)
		# изготовление даты из года и месяца, первого дня месяца и 12:00 по 
		умолчанию
		data.sec$dates <- ISOdate(data.sec$year, data.sec$month, 1)
		# получение правильного дня из дня который вышел за границы месяца
		data.sec$dates <- data.sec$dates + (as.integer(data.sec$day) - 1) * 60*60*24
		# установка 00:00  
		data.sec$dates <- data.sec$dates - 60*60*12 
		# установка вермени по часам прибора
		data.sec$dates <- data.sec$dates + parse_time(data.sec$time)
		
		\end{lstlisting}
	\end{ListingEnv}
	
	%\lstset{% general command to set parameter(s)
	%	basicstyle=\footnotesize} % print whole listing small
	
	\begin{ListingEnv}[H]	
		\caption{Алгоритм коррекции ухода 
			приборных часов на R}
		
		\label{list:timecor}
		\begin{lstlisting}[language={Renhanced}, ]
		\end{lstlisting}
	\end{ListingEnv}
	\begin{lstlisting}[language={Renhanced}, ]
	\end{ListingEnv}
	\begin{lstlisting}[language={Renhanced}, ]
	# time correction------------------------------------------------------------
	
	data.sec <- data.sec%>%
	mutate(dates.UTC = data.sec$dates  - 60*60*3 )
	
	data.sec <- data.sec[,(-17:-22)]
	
	# константа постоянного ухода часов прибора
	kt = (56.77315002) /86400
	
	# вычитание постоянного ухода часов прибора
	data.sec <- data.sec %>%
	mutate(dates.correct.benghin =  dates.UTC - ceiling(
	kt* (dates.UTC - min(dates.UTC))
	))
	
	# восстановление времени начала записи в файл
	data.sec$timestamp.start <-gsub("depron-","0x",data.sec$filename)
	data.sec$timestamp.start <-gsub(".dat","",data.sec$timestamp.start)
	data.sec$timestamp.start <- as.POSIXct(as.integer(data.sec$timestamp.start), 
	origin="1970-01-01", 'GMT' )
	
	# восстановление времни последней записи в файл
	data.sec$timestamp.end <- 
	as.POSIXct(strptime(data.sec$timestamp,format="%d.%m.%Y %H:%M"))
	
	# получени разности между началом файла и горизонтальным приборным временем
	data.sec$time.delta.file.start <- as.numeric(data.sec$dates.correct.benghin - 
	data.sec$timestamp.start ,
	units = "secs") 
	
	data.sec <- data.sec %>%
	group_by(filename) %>%
	distinct(filename) %>%
	summarise(delta.minimum = min(time.delta.file.start)) %>%
	left_join(data.sec, ., by = 'filename')
	
	# table(data.sec$delta.minimum)
	data.sec$time.correct.zolotarev <- data.sec$dates.correct.benghin - 
	data.sec$delta.minimum 
	
	# отбор перескоков времени в приборе более 120 с - меньшие значения возможны 
	при нормальной работе, 
	# большие только при отключениях питания
	data.sec <-mutate(data.sec, lag.delta = delta.minimum - lag(delta.minimum))
	table(data.sec$lag.delta)
	data.sec.switches <-filter(data.sec, abs(lag.delta) >120)
	
	data.sec <-  data.sec %>%
	mutate(switches = cut(data.sec$dates.UTC, 
	breaks = c(min(data.sec$dates.UTC),
	data.sec.switches$dates.UTC,
	max(data.sec$dates.UTC) )))
	
	# xy1 <- xyplot( delta.minimum + switches ~ timestamp.start , data = data.sec,
	#                type = c("o","g"))
	# plot(xy1)
	
	# plot(table(data.sec$delta.minimum))
	# table(data.sec$delta.minimum)
	# median(data.sec$delta.minimum)
	# mfv(data.sec$delta.minimum)
	
	library('modeest')
	if(nrow(data.sec.switches)>0){
	data.sec <-  data.sec %>%
	group_by(switches)  %>%
	mutate(mfv.delta =max(mfv(delta.minimum)))
	}
	if(nrow(data.sec.switches)== 0){
	data.sec <-  data.sec %>%
	mutate(mfv.delta =max(mfv(delta.minimum)))
	}
	
	data.sec <-  data.sec %>%
	mutate(dates.correct = dates.correct.benghin - mfv.delta)
	
	# минус минута так как данные приходят по окончании минуты
	data.sec$dates.correct <- as.POSIXct(data.sec$dates.correct, 'GMT') - 59
	
	data.sec$dates.correct.copy <- as.POSIXct(data.sec$dates.correct, 'GMT')
	# последняя проверка
	# получениЕ разности между началом файла и правильным временем
	data.sec$correct.time.delta.file.start <- as.numeric(data.sec$dates.correct - 
	data.sec$timestamp.start, units = "secs")
	\end{lstlisting}
	
	%
	%Листинг~\ref{list:external1} подгружается из внешнего файла. Приходится 
	%загружать без окружения дополнительного. Иначе по страницам не переносится.
	%
	%\lstinputlisting[firstline=133,lastline=178,language={R},
	%caption={Листинг из 
	%	внешнего файла}]{./listings/save_depron_data.R}
	%
	%\lstinputlisting[firstline=179,lastline=318,language={Renhanced},inputencoding=cp1251,
	%caption={Листинг из 
	%внешнего файла},label={list:external1}]{./listings/save_depron_data.R}
	
	
	
	
	
\end{document}