\chapter{Обработка информации с прибора} \label{chapt3}

Отладка программного обеспечения и проверка обработки сигналов от детекторов на этапе  конструкторских работ производилась подключением канала RS232 к последовательному (COM) порту персонального компьютера. Программное обеспечение контроллера формирует отладочные посылки и массивы тестовых данных и отправляет по интерфейсу USART, реализованному на всех использовавшихся контроллерах. Такой способ передачи тестовых данных выбран как максимально приближенный к условиям реального функционирования прибора. 

Отладка программного обеспечения и проверка обработки сигналов от детекторов на этапе  конструкторских работ производилась подключением канала RS232 к последовательному (COM) порту персонального компьютера. Программное обеспечение контроллера формирует отладочные посылки и массивы тестовых данных и отправляет по интерфейсу USART, реализованному на всех использовавшихся контроллерах. Такой способ передачи тестовых данных выбран как максимально приближенный к условиям реального функционирования прибора. 

Отработка работы прибора в комплексе научной аппаратуры позволяет использовать штатный способ передачи информации по каналу CAN, в таком случае критерием работы прибора является выдача от БИ содержательных блоков информации с меткой, соответствующей прибору ДЭПРОН. 

\section{Общая схема обработки и распределения потоков информации}\label{sec3.1}

Для обработки данных и отладки работы прибора ДЭПРОН были использованы специально разработанные программные средства. Поскольку отладка прибора ДЭПРОН производится подключением по каналу RS232 а при работе в штатном режиме передача данных ведется по каналу CAN, для взаимодействия с прибором были написаны две программы.

\subsection{Программа Depron Terminal}

Данная программа предназначена для отладки прибора во время лабораторных испытаний, проверки работоспособности прибора при приемо-сдаточных работах. Программа была написана в средстве разработки ПО Microsoft Visual Studio на языке c\# c использованием фрэймворка .NET3.5. Пользовательский интерфейс программы построен на основе WinForms, поэтому отличается консервативностью и достаточно низкими аппаратными требованиями.

Программа позволяет:


\begin{itemize}
	\item 	Подключаться к прибору ДЭПРОН по каналу RS232 (с использованием COM порта)
	
	
	\item 	Принимать и отображать тестовые данные сформированные прибором ДЭПРОН
	
	
	\item 	Сохранять запись потока данных на жесткий диск ПК (в фоновом режиме и по запросу)
	
	
	\item 	Открывать сохраненные данные с носителя информации
	
	
	\item 	Посылать команды на прибор ДЭПРОН (в том числе с заданной периодичностью)
	
	
\end{itemize}

\todo{Рисунок. Интерфейс программы \textbf{Depron Terminal}}

Данная программа была использована как основа для разработки отладочной программы для дозиметрических блоков ДБ-8м. Основные принципы работы новой программы, названной \textbf{DB8m Terminal} были сохранены и она обеспечивает те же базовые функции. Дополнительно программа обеспечивает возможность накопления спектров энерговыделения по детекторам ДБ-8м и отображения их в графическом виде, в режиме реального времени. 

\todo{Рисунок. Интерфейс программы \textbf{DB8mTerminal}}


Графическое отображение спектров реализовано с использованием компонента ZedGraph. Введение такой возможности значительно ускорило калибровку и градуировку прибора на источниках радиационного излучения, так что может быть рекомендовано для программ аналогичной направленности.

\subsubsection*{\textbf{Программа DepronExplorerView}}

Данная программа предназначена для просмотра и обработки данных прибора полученных во время комплексных испытаний или во время штатной работы прибора. Аналогично Depron Terminal, данная программа была написана в средстве разработки ПО Microsoft Visual Studio на языке c\# c использованием фрэймворка .NET3.5. Пользовательский интерфейс программы построен на основе WPF.


На момент комплексных испытаний прибора ДЭПРОН программа DepronExplorerView позволяет отображать все типы бинарных данных, полученных от прибора ДЭПРОН, в таблично-текстовой форме и сохранять полученные данные в текстовые файлы. Для удобства использования интерфейс программы выполнен в стиле файлового менеджера.

\subsection*{Структура массивов (базы данных) результатов измерений}

Результаты измерений прибора ДЭПРОН формируется в массивы информации размером 512 байт.


Каждое сообщение состоит из следующих полей:


-  начало сообщения;


-  категория;


-  длина сообщения;


-  данные;


Поле {``}начало сообщения'' содержит 2 байта:


-  байт DLE -- 11110000;


-  байт STX -- 11111111;


На момент написания в программе ДЭПРОН используются нестандартные значения для байт DLE и STX, поэтому во избежание путаницы в дальнейших версия ПО ДЭПРОН будут использоваться общепринятые значения этих байт.


Поле ``категория'' состоит из одного байта (CAT). При обмене с БИ используются варианты сообщений: A, S, H, N. Коды сообщений соответствуют таблице ASCII: A - 01000001, S - 01010011, H - 01001000, N -- 01001110.

Поле ``длина сообщения'' содержит 1 байт (LEN) по умолчанию передается ``\backslash 0'', что означает общую длину посылки 512 байт.


В ином случае значение длины равно общему числу байт сообщения, исключая поле ``начало сообщения''.


Поле ``данные'' (RECORD) содержит данные в соответствии с описанием передаваемых сообщений и их спецификацией.





Общая структура сообщений выглядит следующим образом:


{\small Начало сообщения (DLE,STX)Категория(CAT)Длина(LEN)Данные(RECORD)Метка 1Метка 22 байта2 байта508 байт}

\subsection*{	\textbf{Содержание блоков данных ДЭПРОН}}

Прибор ДЭПРОН в процессе штатной работы формирует несколько типов массивов информации, которые соответствуют различным типам измерений:


\begin{itemize}
	\item 	дозиметрические измерения потока ионизирующих излучений;
	
	
	\item 	измерения спектров потока ионизирующих излучений;
	
	
	\item 	запись данных высокоэнергетичных событий в детекторах;
	
	
	\item 	измерение временного характера кратковременных нейтронных явлений;
	
	
\end{itemize}
Также прибор ДЭПРОН формирует ответ на пришедшую команду от БИ.





Типы массивов данных прибора ДЭПРОН:


\begin{itemize}
	\item 	блок данных ДЭПРОН  A  		Collected dose and fluxes values
	
	
	\item 	блок данных ДЭПРОН  S 		Energy deposition spectra
	
	
	\item 	блок данных ДЭПРОН  H  		High Amplitude Data
	
	
	\item 	блок данных ДЭПРОН  N		Neutron burst data
	
	
	\item 	блок данных ДЭПРОН  Т		квитанция на полученную команду
	

