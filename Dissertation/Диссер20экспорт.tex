%% ================================================================================
%% This LaTeX file was created by AbiWord.                                         
%% AbiWord is a free, Open Source word processor.                                  
%% More information about AbiWord is available at http://www.abisource.com/        
%% ================================================================================

\documentclass[a4paper,portrait,12pt]{article}

\usepackage[T1,T2A]{fontenc}
\usepackage[utf8]{inputenc}
\usepackage[english,bulgarian,russian]{babel}

\usepackage{calc}
\usepackage{setspace}
\usepackage{fixltx2e}
\usepackage{graphicx}
\usepackage{multicol}
\usepackage[normalem]{ulem}

\usepackage{amsmath,amsthm,amssymb}
\usepackage{mathtext}


\usepackage{color}
\usepackage{hyperref}
 
\begin{document}

\setlength{\oddsidemargin}{0.5903in-1in}
\setlength{\textwidth}{\paperwidth - 1.1812in-0.5903in}

\begin{center}
{\large    }
\end{center}


\begin{center}

\end{center}














\begin{center}
{\large На правах рукописи}
\end{center}





\begin{center}
\textbf{{\large Брильков Иван Анатольевич}}
\end{center}





\hypertarget{OLE_LINK62}{\hypertarget{OLE_LINK24}{\hypertarget{OLE_LINK25}{\hypertarget{OLE_LINK61}{\textbf{\colorbox[rgb]{1.000,1.000,0.000}{{\LARGE ОПРЕДЕЛЕНИЕ РАДИАЦИОННОЙ НАГРУЗКИ В КОСМИЧЕСКОМ АППАРАТЕ ПРИ ПОЛЕТЕ ПО ВЫСОКОШИРОТНОЙ ОРБИТЕ}}}}}


}}


\begin{center}
{\large Специальность:}
\end{center}


\begin{center}
\hypertarget{OLE_LINK53}{\hypertarget{OLE_LINK54}{{\large 05.26.02}}}{\large . -- }\hypertarget{OLE_LINK55}{\hypertarget{OLE_LINK56}{{\large Безопасность в чрезвычайных ситуациях }
\end{center}


\begin{center}
{\large (авиационная и ракетно-космическая техника)}}}
\end{center}


\begin{center}
{\large Диссертация на соискание ученой степени }
\end{center}


\begin{center}
{\large кандидата }\colorbox[rgb]{1.000,1.000,0.000}{{\large технических наук}}
\end{center}


\begin{center}
{\large Научный руководитель:}
\end{center}


\begin{center}
\hypertarget{OLE_LINK59}{\hypertarget{OLE_LINK60}{{\large кандидат физико-математических наук}}}{\large ,}
\end{center}


\begin{center}
\hypertarget{OLE_LINK57}{\hypertarget{OLE_LINK58}{{\large Бенгин Виктор Владимирович}}}{\large  }
\end{center}


\begin{center}

\end{center}





\begin{center}
\colorbox[rgb]{1.000,1.000,0.000}{{\large Москва, 2010 г.}}
\end{center}


\section*{}



\setlength{\oddsidemargin}{0.5903in-1in}
\setlength{\textwidth}{\paperwidth - 1.1812in-0.5903in}

\textbf{{\large \newpage
}}


\section*{}

\textbf{{\large Оглавление}}







\tableofcontents 















\setlength{\oddsidemargin}{0.5903in-1in}
\setlength{\textwidth}{\paperwidth - 1.1812in-0.5903in}

\newpage



\section*{\textbf{Список сокращений}}













\newpage



\section*{\textbf{Введение}}

\subsection*{\textbf{Актуальность работы}}

\hypertarget{OLE_LINK71}{\hypertarget{OLE_LINK70}{\hypertarget{OLE_LINK64}{\hypertarget{OLE_LINK63}{Актуальность темы связана с имеющимися планами создания пилотируемого космического аппарата, работающего на высокоширотной орбите, поэтому необходимо провести дозиметрический мониторинг области околоземного пространства, в которой планируется проведение перспективных пилотируемых полетов. Работа направлена на осуществление такого мониторинга.}}


\subsection*{\textbf{Содержание работы}}

\hypertarget{OLE_LINK72}{\hypertarget{OLE_LINK66}{\hypertarget{OLE_LINK65}{\hypertarget{OLE_LINK73}{Диссертационная работа посвящена исследованию распределения мощности дозы космической радиации на высокоширотной траектории.}} 


Прибор \textbf{ДЭПРОН} (\textbf{Д}озиметр \textbf{Э}лектронов, \textbf{ПРО}тонов, \textbf{Н}ейтронов) предназначен для измерения поглощенных доз и спектров линейной передачи энергии от высокоэнергичных электронов, протонов и ядер космического излучения, а также регистрации потоков тепловых и медленных нейтронов.


\subsection*{\textbf{Цель работы}}

\hypertarget{OLE_LINK67}{Экспериментальное определение радиационной нагрузки при полете по полярной орбите с помощью дозиметрической аппаратуры, разработанной для этих целей. 


\colorbox[rgb]{1.000,1.000,0.000}{Целью настоящей диссертационной работы является }\hypertarget{OLE_LINK68}{\hypertarget{OLE_LINK69}{\colorbox[rgb]{1.000,1.000,0.000}{исследование распределения мощности дозы космической радиации на высокоширотной траектории на фазе роста 24-го цикла солнечной активности}.}}}


\subsection*{\textbf{Задачи работы}}

Систематизация и обобщение характеристик радиационных условий на высокоширотных орбитах (аналогичных орбитам аппаратов БИОН, Прогноз, Cluster) для разработки программы эксперимента.


Разработка и изготовление на основе полупроводникового детектора ионизирующих излучений бортового дозиметра для определения эквивалентной дозы космического излучения.


Подготовка и проведение эксперимента с дозиметром на борту КА «Ломоносов», обработка и анализ информации.








\subsection*{\textbf{Научная значимость и новизна работы}}

Исторически сложилось, что наиболее изученными в дозиметрическом плане являются обиты с небольшим и средним наклонением, соответствующим широте космодрома запуска космического аппарата. В частности, наиболее изученной является орбита с наклонением порядка 50 градусов МКС. В связи с увеличением интереса к запускам с космодрома «Восточный» (имеющим географические координаты 51°45 с. ш. 128°07 в. д.) а также с космодрома «Плесецк», являющимся на данный момент самым северным из действующих космодромов в мире (координаты 63°00 с. ш. 41°00 в. д.), появляется необходимость в экспериментальных данных дозиметрических измерений на высокоширотных орбитах. 


\subsection*{\textbf{Практическая ценность работы}}

Имеющиеся планы создания пилотируемого аппарата предназначенного для полетов по орбитам с наклонением до 72° приводят к необходимости более детального описания распределения мощности дозы при полетах по орбитам данного класса. Необходимо экспериментально проверить имеющиеся модели расчета дозиметрических характеристик поля излучения на трассе перспективного ПКА, в том числе методы определения эквивалентной дозы.


Планы разработки нового объекта «Ломоносов», траектория полета которого позволяет провести опережающее дозиметрическое зондирование трассы перспективного ПКА в том числе высокоширотных участков траектории, создают необходимые предпосылки для проведения измерений, но для этого КА «Ломоносов-300» требуется оснастить соответствующей дозиметрической аппаратурой.


Так как в настоящее время в России отсутствуют собственные разработки для измерения эквивалентной дозы космического излучения, создание дозиметрического средства для использования в космосе, работающего в активном режиме, адаптированного к работе на высокоширотных орбитах и пригодного для установки на «Ломоносов-300» (а также в перспективе на другие КА), является актуальной технической задачей.


\subsection*{\textbf{Личный вклад автора}}

Автором проведены тепловые испытания полупроводникового детектора с усилительным трактом будущего прибора. Проведены проверки работоспособности усилительного тракта с радиоактивными источниками ОСГИ Cs137 и Co60. Проведены испытания детекторов тепловых нейтронов на лабораторном источнике нейтронов. Проведены работы по стыковке и согласованию платы цифровой обработки сигналов с аналоговыми усилительными трактами и дискриминирующими блоками прибора. Написана программа на С++ для контроллера платы цифровой обработки сигналов. Для наземной отработки и испытаний написана программа для ПК на WinForms/С\#, позволяющая оперативно контролировать параметры работы прибора и выходные данные. Написана программа на базе пакета Geant4 для математического моделирования характеристик прибора Дэпрон, а также программы для моделирования характеристик приборов ДБ-8 и Liulin.





\subsection*{\textbf{Основные положения, выносимые на защиту}}

\subsection*{\textbf{Апробация работы}}










\newpage



\section*{\textbf{Глава 1 Обзор Литературы}}

\subsection*{	\textbf{Радиационная обстановка на высокоширотных околоземных орбитах. Вопросы, требующие детального исследования.}}

Исследования радиационной обстановки в космическом пространстве связано с началом полетов автоматических аппаратов и человека в космос.  Широкое распространение технологий, связанных с использованием космической техники, а также непрерывные пребывание человека в космическом пространстве во время миссий на космических станциях МИР и МКС позволило выявить ряд опасностей космических полетов, среди которых особое внимание следует уделить радиационной опасности [Логачев Ю.И. 2007].


Запуск 2-го и 3-го спутников Земли, с приборами, изготовленными в НИИЯФ МГУ,  показал принципиальную возможность полета человека в космос,  однако, как можно заметить из данных полученных при начальных исследованиях радиационной обстановки, на орбите земли существуют отдельные области повышения радиационного фона (Рисунок \ldots{}). Существование данных областей связано с неоднородностями магнитного поля Земли и приводит к формированию области повышения потоков частиц в Южно Атлантической области, названной Южно-Атлантической Аномалией (ЮАА) [Вернов С.Н., Савенко И.А., Шаврин П.И., Писаренко Н.Ф.\emph{ Обнаружение внутреннего радиационного пояса на высоте 320 км в районе южно-атлантической магнитной аномалии. ДАН СССР, 1961, т. 140, N 5, c. 1041-1044}]. В первом приближении для описания магнитного поля  Земли на высотах до 2000 км можно использовать представление модели смещенного диполя, этот подход позволяет учитывать ЮАА [Модель космоса 3 том 20стр].





Рисунок Распределение потоков частиц по данным 2-го корабля-спутника над поверхностью земного шара на высоте 320 км. (цифpы у линий дают потоки частиц в см\textsuperscript{-2} c\textsuperscript{-1}) [Логачев Ю.И. 2007]


Таким образом, магнитное поле Земли экранирует космические аппараты, находящиеся на средних широтах и невысоких орбитах порядка трехсот-четырехсот километров от поверхности Земли (именно на этих высотах поддерживается обращение космических станций). Значительный вклад, до 60\%,  в дозовую нагрузку аппараты и их экипаж получают в ЮАА [?].


Другими, важными с точки зрения радиационной обстановки, являются приполярные области [Горчаков Е.В. \textbf{Внешний радиационный пояс и полярные сияния. }\emph{Искусственные спутники Земли, 1961, вып. 9, с. 66-70.}].


При выборе более высокоширотных и более высоких орбит дополнительного внимания требуют области полярных шапок, так как в этих областях границы радиационных поясов Земли ближе к поверхности. Даже на небольших высотах, начиная от 300 км в интервале геомагнитных широт 55-70 наблюдается резкое возрастание интенсивности излечения и частицами, составляющими этот внешний радиационный пояс являются электроны различных энергий, их поток достигает 10\textsuperscript{5} см\textsuperscript{-2} сек\textsuperscript{-1} стер\textsuperscript{-1} [Исследования космических лучей и земного корпускулярного излучения при полетах ракет и спутников; УФН, т. 70, вып. 4, 585 (1960)]. При солнечных событиях в этих областях создаются условия для многократного повышения потоков частиц, что может привести к необходимости специальных мер по предотвращению чрезмерной радиационной нагрузке на экипаж космического аппарата.

















{\large \newpage
}


\subsection*{	\textbf{Методы регистрации доз}\textbf{ }}

Среди методов регистрации ионизирующих излучений можно выделить несколько наиболее используемых:


Газовые ионизационные детекторы, в том числе пропорциональные и газоразрядные счетчики.


Сцинтилляционные детекторы


Полупроводниковые детекторы 


Трековые детекторы


Спектрометры заряженных частиц, спектрометры нейтронов и спектрометры 


Первые эксперименты в космосе по измерению радиационных условий предполагали использование ионизационных камер достаточно большого размера (десятки см\textsuperscript{3}) однако 


\colorbox[rgb]{1.000,0.000,0.000}{Дописать}


\subsection*{	\textbf{Приборы, использовавшиеся для дозиметрического контроля в космосе}}

\subsection*{	\textbf{Пассивные дозиметры}}




\colorbox[rgb]{1.000,0.000,0.000}{Дописать}


CR-39 тканеэквивалентный трековый детектор [Zhou, D., O'Sullivan, D., Semones, E., et al. Radiation dosimetry or high LET particles in low Earth orbit. Acta Astranautica 63, 855--864, 2008]


TLD-100, -600, -700, OSLD Люминисцентные детекторы [?]


BR\&Bya type NPE FILM фотографическая эмульсия


Pille портативная считывающая система[(Apathy et al., 2002, Apathy et al., 2007]


EVARM детектор MOSFET 


Матрешка-Р ионизационная камера [Machrafi, R., Garrow, K., Ing, H., et al. Neutron dose study with bubble detectors aboard the International Space Station as part of the MATROSHKA-R experiment. Radiation Protection Dosimetry 133 (4), 200--207, 2009]


\subsection*{	\textbf{Активные детекторы}}

\subsection*{	\textbf{Liulin-4}}

Детекторы серии Liulin используются с 1988 года, когда их первое поколение было использовано на борту космической станции МИР [Caffrey JA 2011]. Liulin-4 не последний прибор в этой серии, но его простое устройство и компактные размеры обеспечивают удобство использования для многих конкретных задач. Этот спектрометр состоит из единственного кремниевого детектора, зарадочувствительного предусилителя, микроконтроллера и флэш-памяти. Насыщенный литием кремниевый детектор имеет толщину 0,3 мм и площадь 2 см\textsuperscript{2}. В приборе установлен 12-ти битный АЦП, но только 8 бит из них используется для получения 256 канального спектра энерговыделения за выбранный интервал времени накопления: от 10 до 3539 с. Амплитуда импульса определяется после предусилителя и разделяется по 256 энергетическим каналам, начинающимся с 0,02 МэВ до 20 МэВ. Выделение энергии, большее 20 МэВ записывается в наибольший энергетический канал [Dachev, Ts.,2002] .


Для определения дозы в данном типе детектора энерговыделение в каждом канале определяется умножением счета в детекторе на энергию канала. Эти результаты делятся на массу объема детектора и суммируются для определения общей дозы по всем каналам [Dachev, Ts.,2002]. Записанная форма спектра энерговыделения может предоставить дополнительную информацию относительно природы доминирующего радиационного поля (ЮАА, ГКЛ и др. ), но не является достаточно подробной для определения ЛПЭ воздействующей радиации [Caffrey JA 2011]. 


Размер и портативность спектрометра типа Liulin-4 делает его жизнеспособным кандидатом для активной персональной дозиметрии во время солнечного события, но ограничения в возможности определения эффективной ЛПЭ и эквивалентной дозы предотвращают вытеснение методов пассивной дозиметрии. Liulin-4 существует во многих модификациях и с многими опциями и может работать как на химическом источнике тока, так и на непрерывном питании, функционировать как с внешним ЖК-дисплеем так и без него, и может включать GPS-приемник [Dachev, Ts., Tomov, B., Matviichuk, Yu., et al. Calibration results obtained with Liulin-4 type dosimeters. Advances in Space Research 30 (4), 917--925, 2002.].	


\subsection*{	\textbf{DOSTEL полупроводниковый телескоп}}

DOSTEL -- Дозиметрический телескоп был разработан в 1995 году как малый телескоп частиц для использования на миссиях космических шаттлов к космической станции МИР. Прибор включает в себя два кремниевых детектора по технологии PIPS, расположенных как телескоп [Beaujean, R., и др. 2002]. Каждый детектор имеет толщину 0,315 мм с чувствительной зоной 6,93 см2, зазор в 15 мм между детекторами дает геометрический фактор 824 см\textsuperscript{2}ср (единица измерения определяется чувствительной площадью детектора и полем зрения) для детектирования совпадающих событий [Beaujean, R., и др. 2002]. Каждый детектор соединен с зарядочувствительным усилителем через интегрирующую емкость, двух стадийным усилителем импульсов, двумя пиковыми детекторами, двумя RC-фильтрами для снижения уровня шумов и 8-ми битным АЦП. Такая компоновка позволяет поводить анализ амплитуд импульсов отдельно для высокого и для низкого энергетического диапазона [Beaujean, R., и др. 2002].


Когда совпадающее событие записано обоими детекторами, становится возможным определить ЛПЭ падающего излучения. Так как известно, что траектория частицы ограничена конусом возможных направлений, средняя толщина детектора может быть использована для оценки длины трека частицы. Делением энерговыделения на среднюю длину свободного пробега, для данного детектора 0,364 мм [Beaujean, R., и др. 2002] с плотностью 2,33 г/см\textsuperscript{3} [Knoll GF, Radiation detection and measurement, third edition, Wiley: 2000, p802  на странице 357]  можно получить приближенное значение ЛПЭ. Результат таких вычислений нормируется на известный коэффициент для перехода от ЛПЭ кремнии к ЛПЭ в воде, таким образом прибор DOSTEL записывает ЛПЭ в диапазоне от 0,1 до 240 кэВ/мкм [Beaujean, R., Kopp, J., Burmeister, S., et al. Dosimetry inside MIR station using a silicon detector telescope (DOSTEL). Radiation Measurements 35, 433--438, 2002].





\colorbox[rgb]{1.000,0.000,0.000}{Дописать}\colorbox[rgb]{1.000,0.000,0.000}{:}


RRMD-III Determines path length with PSDs [Doke et al. (2001, 2004)]


Liulin-5 Assumes mean-chord-length across FOV in LET calculation [Semkova et al. (2004, 2007)]


Liulin Phobos Assumes mean-chord-length; orthogonal telescopes [Dachev et al. (2009)]


CPDS Determines path length with PSDs; can determine species for C, N, and O particles [Lee et al. (2007)]


TEPC Assumes mean-chord-length for all angles; LET assumed equal to y (Lineal energy) [Badhwar et al. (1996), Gersey et al.(2002, 2007)]


R-16 Pulse-type ion chamber: 1 pulse per 5 mrad; shallow and deep dose rates; assumes average LET [Mitricas et al. (2002), Badhwar (2000)]


BBND Heavy system; short-term experiment; requires 3He Koshiishi et al. (2007), Matsumoto


et al. (2001)





\subsection*{	\textbf{ Математическое моделирование дозиметрических проборов для космических условий}}

Математическое моделирование широко применяется на всех этапах создания исследовательских проборов предназначенных для использования в условиях космоса. В первую очередь оно необходимо на этапе проектирования аппаратуры для выбора характеристик регистрирующих радиацию модулей исходя из поставленных экспериментальных задач [{\footnotesize Luszik-Bhadra M. }{\footnotesize et}{\footnotesize . }{\footnotesize al}{\footnotesize .}{\footnotesize ,}{\footnotesize  2009, }{\footnotesize Hassler}{\footnotesize  }{\footnotesize D}{\footnotesize   }{\footnotesize et}{\footnotesize . }{\footnotesize al}{\footnotesize .}{\footnotesize ,}{\footnotesize  2008}{\footnotesize ]}. На последующих шагах разработки аппаратуры математические методы используются при верификации результатов калибровочных и градуировочных испытаний на источниках ИИ и ускорителях заряженных частиц [Zeitlin C. {\footnotesize et}{\footnotesize . }{\footnotesize al}{\footnotesize .}{\footnotesize ,}{\footnotesize  20}{\footnotesize 10}]. Также одним из основных применений является уточнение функции отклика прибора во время штатной работы [C. Zeitlin {\footnotesize et}{\footnotesize . }{\footnotesize al}{\footnotesize .}{\footnotesize ,}{\footnotesize  20}{\footnotesize 10}]. 


Среди математических методов моделирования взаимодействия ИИ и нейтральных излечений с материалами и детектирующими модулями приборов следует отметить наиболее используемые программные пакеты, основанные на методе Монте-Карло:


	GEANT4


	FLUKA


	SHIELD


	PHITS





\subsection*{	\textbf{GEANT4}}

Данная система математического моделирования разрабатывается для нужд работы ЦЕРН и активно используется в ряде областей науки, медицины и технологии.


\subsection*{	\textbf{FLUKA}}




\subsection*{	\textbf{SHIELD}}
















\subsection*{	\textbf{ }\textbf{Возможности КА Ломоносов в продолжении ряда российских исследований радиационной обстановки}}

На каждом российском пилотируемом корабле со времен первого полета человека в космос устанавливались дозиметрические приборы, изготовленные в НИИЯФ МГУ, полный список и результаты этих экспериментов можно найти в монографии Ю.И. Логачева 2007г.

















\newpage



\section*{\textbf{Глава 2 Аппаратура для проведения исследований}}

\subsection*{	\textbf{ }\textbf{Прибор Дэпрон}}

В состав прибора ДЭПРОН входят два узла с полупроводниковыми детекторами и два узла с газоразрядными гелиевыми счетчиками нейтронов. Также в состав прибора входят узлы усиления и формирования сигналов от полупроводниковых и нейтронных детекторов и узел цифровой обработки сигналов.


Для регистрации поглощенной дозы используются узлы с полупроводниковыми детекторами. Для получения информации о величине поглощенной дозы используется принцип регистрации величины заряда в объеме полупроводника, пропорционального энерговыделению в данном объеме. 


Оба полупроводниковых детектора и скомпонованы в кассету и расположены в относительной близости друг от друга. Схема построения прибора с расположением двух полупроводниковых детекторов параллельно была использована для получения информации о ЛПЭ частиц, прошедших одновременно оба детектора. 


 





Рисунок Блок-схема прибора ДЭПРОН


\newpage



\subsection*{	\textbf{ }\textbf{Конструкция}}

Прибор состоит из одного блока, габаритный чертеж которого представлен в Приложении 1. Габаритные размеры прибора: длина  280 мм, ширина 160 мм, высота 78 мм. Масса прибора - 3 кг. Корпус прибора составлен из шести пластин Д16т -- листового дюралюминия, толщиной 4,5 мм, обработанного на станке ЧПУ. В каждой пластине фрезерованы повторяющиеся выборки треугольной формы до толщины 2 мм. Выборки расположены таким образом, чтобы сформировать «ребра» жесткости в стенках прибора. С лицевой стороны пластины корпуса оксидированы, с целью получения электропроводной поверхности всего прибора.


\begin{center}

\end{center}


\begin{center}
Рисунок Вариант размещения выборок в днище прибора ДЭПРОН. В последствии данный вариант переработан и заменен исходя из конструктивных соображений крепления модулей электроники и улучшения теплосброса источников питания, через термоконтакт с бортом КА.
\end{center}


На лицевом торце прибора распложены два разъема СНП-333, используемых для передачи данных в БИ аппаратуры спутника (разъем Х1) и для передачи питания в прибор ДЭПРОН от бортовой аппаратуры спутника (разъем Х2). Также на лицевой панели находятся два разъема РС-7 предназначенные для передачи информации по каналу РС232 от прибора ИМИСС-1 (разъем Х5) и сквозной передачи питания от бортовой аппаратуры к прибору ИМИСС-1 (разъем Х4). Во всех перечисленных разъемах предусмотрен контроль стыковки разъемов с помощью короткозамкнутых линий, а также дублирование информационных и токонесущих линий.


Дополнительно на лицевую панель прибора вынесен технологический разъем РС 19 ХТ3, используемый для проверки функционирования прибора в лабораторных условиях методом подачи на детекторные узлы калиброванных сигналов с генератора, а также для контроля внутренних рабочих напряжений. Проверка работоспособности прибора и подача сигналов с генератора осуществляется с помощью  блока КПА, имеющему четыре экранированных канала для передачи низкоамплитудных сигналов и два светодиодных индикатора для контроля наличия рабочих напряжений  +5 В и +12 В в приборе ДЭПРОН. В штатном режиме работы данный разъем не подключен и закрыт заглушкой. Схема распределения линий в разъемах представлена в Приложении 2.


Платы электроники блоков усиления и формирования аналоговых сигналов располагаются в трех тонкостенных алюминиевых кассетах и выполнены в формате 11-ти контактных печатных плат размерами 34х50мм. Данный формат печатных плат распространен в производстве научной аппаратуры изготовления НИИЯФ МГУ и с успехом применяется для космической аппаратуры уже на протяжении нескольких десятков лет. Применение данного стандарта позволяет соблюсти принцип модульности построения приборов, используя отработанные в космических условиях надежные схемы, компонуя из них тракты с параметрами, заданными потребностями текущих экспериментальных задач. 





Рисунок Внутренняя компоновка модулей прибора ДЭПРОН. Вид сверху со снятой крышкой прибора. 


В средней части рисунка последовательно располагаются три корпусных кассеты с платами электроники: левая и правая кассеты содержат платы формирователей триггерных сигналов от детекторов, центральная кассета ориентирована перпендикулярно и содержит две платы полупроводниковых детекторов и ЗЧУ, а также платы дополнительного усиления.


В нижней части рисунка находится нейтронный счетчик СИ13Н (циллиндр), экранированный 1 см оргстекла





\subsection*{	\textbf{Детекторы}}

\hypertarget{_Toc309891358}{Дозиметр заряженных частиц выполнен на кремниевых ионно-имплантированных Д1 пролетных детекторах, работающих в режиме регистрации амплитуд импульсов. Детекторы изготовлены по специальному заказу НИИЯФ МГУ в ООО «Детектор-СИ» в соответствии с АБЛК.418219.402ТУ. Детекторы, в соответствии с паспортом, предназначены для спектрометрии и радиометрии заряженных частиц в составе предназначенной для этих целей аппаратуры. Детекторы могут эксплуатироваться при атмосферном давлении или в вакууме до 10\textsuperscript{-6} мм.рт.ст. чувствительный элемент детектора изготовлен из высокоомного кремния n--типа по технологии ионной имплантации.


Значения параметров детекторов приведены в таблице.


Наименование параметраФактические параметрыПримечанияРабочее напряжение, В90Аттестация производилась при 26 CОбратный ток, нА4Энергетический эквивалент шума, кэВ5Постоянная времени квазигауссова формирования импульса, мкс2Предельно допустимое напряжение, В130





Рекомендуемая схема включения детектора приведена на рис. 


\\Рисунок Схема включения детектора.


+Еп -- источник напряжения;


Rсм -- сопротивление смещения;


D1 -- Детектор;





Для обеспечения надежности используются два полупроводниковых детектора. Детекторы образуют телескоп, что обеспечивает возможность регистрировать спектр ионизационных потерь.}


Детектор нейтронов выполнен на счётчике медленных нейтронов «СИ-13Н», представляющем собой газоразрядный  счетчик, работающий в режиме коронного разряда. Для обеспечения надежности используются 2 счетчика. Второй детектор нейтронов окружен замедляющей оболочной из поликарбоната, что позволило расширить энергетический диапазон регистрируемых нейтронов. При прохождении нейтрона через газ Не-3, наполняющий счетчик, происходит ядерная реакция n+3He = p+T+764 КэВ. Продукты реакции вызывают ионизацию газа в счётчике, что приводит к образованию газового разряда и появлению электрического импульса на электроде счетчика. Импульс поступает на вход усилителя-формирователя и, затем, поступает на регистр прерываний процессора, где используется для подсчета числа зарегистрированных нейтронов.





\subsection*{	\textbf{Аналоговая обработка сигналов}}

Платы полупроводниковых детекторов и предусилителей (внутренний номер SSD006) изготовлены методом фотолитографии в стандартном формате 34х50, использование современных миниатюрных электронных компонент позволило совместить блоки предусиления и детектирования на одной плате и закрыть единым экраном от электромагнитных помех.


Сигнал с полупроводникового детектора поступает на зарядовочувствительный предусилителя A225F, фирмы AMPTEC, специализирующейся на производстве компонент для космической промышленности. 





\colorbox[rgb]{1.000,1.000,0.000}{Рисунок A225F}


На выходах предусилителя формируются два сигнала. Один (S-сигнал) - имеет амплитуду пропорциональную заряду, образовавшемуся в детекторе и длительность порядка 5 -- 10 мксек. Этот сигнал поступает на амплитудно-цифровой преобразователь (АЦП). Второй сигнал предусилителя A225F (tсигнал) имеет короткое, менее 0.5 мксек, время задержки от момента прихода сигнала с детектора до максимума амплитуды и используется для запуска процесса цифровой обработки пришедшего импульса. Этот (t-сигнал) сигнал поступает на вход усилителя и, после усиления, поступает на регистр прерываний процессора, где используется для запуска процесса преобразования амплитуды сигнала, поступившего на АЦП, в код. Дальнейшая обработка сигналов с полупроводниковых детекторов производится микропроцессором прибора в цифровой форме.





\subsection*{	\textbf{Цифровая обработка сигналов}}

Для записи результатов измерений прибора используется внутренняя память микроконтроллера, входящего в состав узла цифровой обработки сигналов. В нее записываются, а затем передаются в Блок Информации КА «Ломоносов» кадры информации.


На этапе опытно-конструкторских разработок (при макетировании прибора ДЭПРОН) в качестве узла цифровой обработки сигнала использовался 8-битный микроконтроллер ATmega128. Данная микросхема отличается низкой потребляемой мощностью и обладает развитыми средствами ввода данных и обмена информацией, а также достаточной вычислительной мощностью. Печатная плата контроллера была разработана в  НИИЯФ МГУ Н.Н. Веденькиным и Д.Г. Аксельродом. На плате расположены два АЦП, а также дополнительная память, независимый преобразователь питания и контроллер обеспечения связи по последовательному каналу (RS232). Как показали опытно-конструкторские работы, проведенные с макетом дозиметра ДЭПРОН, данный узел обеспечивает потребности по бортовой обработке сигналов от детектора по производительности, несмотря на то, что по современным меркам частота работы ядра процессора невелика - 16 MHz. Также выбранный контроллер обладает достаточным для поставленной задачи количеством входных каналов.


Для преобразования амплитуды импульсов, сформированных на выходе аналоговых трактов усиления, использовались 12-ти битные АЦП AD7495 фирмы Analog Devices со скоростью работы 1 MSPS (миллион сэмплов в секунду). Данные АЦП используют высокоскоростной последовательный интерфейс (SPI -- Serial Peripheral Interface), который был реализован программным способом. Управление моментом захвата амплитуды входного сигнала также производилось программным способом подачей цифрового сигнала «0» на линию CS. 





Рисунок Использованный в приборе ДЭПРОН режим работы АЦП (AD7495), по материалам:  «1 MSPS,12-Bit ADCs  AD7475/AD7495», One Technology Way, P.O. Box 9106, Norwood, MA 02062-9106, U.S.A., 2005 Analog Devices, Inc.


В первой версии платы цифровой обработки сигналов подключение обоих АЦП к контроллеру прибора производилось по независимым каналам: CS (Chip Select -- Активный логический вход АЦП), SCLK (Serial Clock -- логический вход АЦП), SDATA (Data Output -- логический выход АЦП). Задача максимально быстрого захвата сигналов с выхода предусилителя решалась включением встроенного в АЦП устройства выборки и хранения (англ. {``}track and hold circuit'') в момент получения контроллером сигнала от таймингового выхода предусилителя. Для этого прерывания контроллера настроены при получении такого сигнала на выдачу управляющего сигнала на вход CS АЦП, ответственного за оцифровку сработавшего канала аналоговой части прибора. Дальнейшая оцифровка амплитуды захваченного в буфере АЦП сигнала производилась после выхода из процедуры обработки прерывания, так как этот процесс отнимает значительное время. Испытания процедуры управления оцифровкой АЦП показали, что точность измерений АЦП чувствительна к временной регулярности тактирующего сигнала подаваемого на SCLK АЦП. Одной из причин таких нерегулярностей является возможность срабатывания прерывания в ПО контроллера во время исполнения процедуры генерации тактирующих импульсов, что в условиях эксплуатации прибора при высоких потоках ионизирующих излучений (например, в области ЮАА) не редкость. Временное отключение обработки прерываний может устранить данный недостаток работы прибора, однако испытания такого режима работа показали накопление необработанных прерываний в буфере контроллера, которые впоследствии обрабатывались неверно, из-за чего решено отказаться от использования этого режима.





Рисунок. Осциллограмма: регулярность тактирующего сигнала, подающегося на SCLK АЦП.


Выявление нерегулярности тактирующего сигнала потребовало проверку этого сигнала с помощью осциллографа. Дизассемблирование скомпилированного кода ПО микроконтроллера показало критические места кода, требующие изменения алгоритма генерации тактирующих импульсов и добавления промежутков простоя процессора (\_nop -- в коде {``}no operation''). Окончательная проверка регулярности сигнала, генерируемого выверенным кодом, производилась снятием временной развертки тактирующего сигнала  на осциллографе. Данный подход использовался и при последующих отработках работы АЦП прибора ДЭПРОН.





Рисунок Блок схема подключения АЦП к контроллеру версия 1.


Таким образом, в первой версии платы цифровой обработки сигналов использовались шесть независимых каналов контроллера, что ограничивало возможность подключения дополнительных информационных каналов с детекторной части прибора. Также одой из проблем данного подхода является двойная нагрузка на микроконтроллер прибора ДЭПРОН, так как управляющие сигналы генерируются программным способом. Такой подход предоставлял сомнительное преимущество в независимом управлении АЦП из программного обеспечения микроконтроллера, поэтому было принято решение изменения способа подключения АЦП.


Следующим конструктивным решением было включение обоих АЦП в параллельный режим работы, когда управляющий (CS) и тактирующий (SCLK) сигналы подаются на оба АЦП. Каналы данных (SDATA) подключены к независимым входам контроллера. 





Рисунок Блок схема подключения АЦП к контроллеру версия 2.





При проектировании Блоков обработки Информации (БИ) было принято решение по организации обмена по каналу CAN между дочерними приборами, входящими в Комплекс Научной Аппаратуры (КНА) «Ломоносов». Однако использованный для макетирования контроллер ATmega128, и данный контроллер был заменен на AT90CAN128. Данное решение было продиктовано минимальными изменениями уже разработанного программного обеспечения и незначительными доработками печатных плат, необходимым для внедрения контроллера AT90CAN128.


Опыт работы с данным контроллером также показал его применимость для целей построения полноценного дозиметра ионизирующих излучений. Тем не менее, по требованию других участников проекта данный контроллер был заменен более современным и более производительным контроллером AT91SAM7X256. Всего в составе КНА насчитывается 4 прибора, в которых использована схема цифровой обработки сигнала на базе AT91SAM7X, некоторые из этих приборов испытывали нехватку производительности данного модуля до замены ЦПУ. В целях унификации разработанный аппаратуры модуль цифровой обработки сигналов и связи был заменен и в приборе ДЭПРОН. Данное изменение состава прибора повлекло за собой необходимость повторения цикла разработки программно-математического обеспечения прибора и проведения повторных калибровок АЦП и счетных каналов схемы цифровой обработки. Необходимость данных работ обусловлена принципиальным отличием архитектуры контроллера: в исходном варианте это архитектура AVR, а в окончательном ARM. 


В финальном варианте цифровая обработка сигналов осуществляется с помощью микропроцессора AT91SAM7X512. Программно-математическое обеспечение ДЭПРОН функционирует на одной микропроцессорной плате SSD234. Данная плата собрана на базе микроконтроллера AT91SAM7X512 производства фирмы ATMEL, и содержит процессор ARM7 TDMI® ARM® Thumb® с 32-разрядной RISC-архитектурой команд.


Программное обеспечение процессора осуществляет регистрацию сигналов, поступающих со схем преобразования импульсов с детекторов, их преобразование и накопление, передачу результатов по каналу связи с блоком информации КА. Объем сбрасываемой информации не превышает 1 Мбайт/сутки.





\subsection*{	\textbf{Связь с }\textbf{внешними системами}}

\emph{Связь с Блоком Информации «Ломоносов»}


Связь с БИ осуществляется посредством канала Controller Area Network (CAN), использующегося в качестве стандарта промышленных сетей. CAN ориентированн на объединение в единую сеть различных исполнительных устройств и датчиков. Режим передачи данных - последовательный, широковещательный, пакетный. Программные модули и аппаратные схемы разрабатывались для комплекса аппаратуры в целом Н.Н. Веденькиным и прошли проверку при доводке аппаратуры и комплексных испытаниях КНА.


Прибор ДЭПРОН формирует в рабочем режиме пакеты данных по 512 байт, которые накапливаются во внутренней памяти контроллера. Подготовленная очередь пакетов  передается на БИ КА «Ломоносов», где накапливается для передачи на Землю.


Передача информации от космического аппарата происходит через сеть фиксированной спутниковой связи. Данные передаются через общественную сеть Интернет и архивируются на специально выделенном сервере данных. Альтернативно, при отсутствии подключения к спутниковой сети связи, используется канал передачи телеметрической информации с платформы КА «Ломоносов», при таком подключении данные ДЭПРОН поступают на Землю через центр управления полетами (ЦУП) и ввиду ограниченной пропускной способности этого канала данные передаются частично.


\emph{Связь с прибором ИМИСС}


Связь с прибором ИМИСС-1 осуществляется по каналу РС232 (USART). Поступающая информация транслируется прибором ДЭПРОН в БИ по каналу CAN без изменений. В соответствии с расчетным объемом данных от прибора ИМИСС затраты производительности микроконтроллера ДЭПРОН на трансляцию данных в БИ будут незначительны по отношению к затратам на выполнение основных задач прибора ДЭПРОН.





\subsection*{	\textbf{Питание}}

Электропитание схем прибора ДЭПРОН осуществляется с использованием DC/DC преобразователей. Напряжение питание бортовой сети 27В, подключено через разъем Х2 прибора ДЭПРОН и поступает на два преобразователя 28/12 В. С первого преобразователя напряжение поступает на стабилизатор напряжения и далее из этого напряжения формируются номиналы: +6 В, для питания схем усилителей, формирователей и микропроцессора. Со второго преобразователя питание поступает на преобразователь +70 В, для питания полупроводниковых детекторов и на преобразователь +1200 В, для питания газоразрядных счетчиков.





\subsection*{	\textbf{ }\textbf{Программное обеспечение}}

Программно-математическое обеспечение прибора ДЭПРОН состоит из программы для контроллера прибора, написанной на языке C++(C) c использованием пакета IAR  Workbench® для микроконтроллеров архитектуры ARM®. 


Исполняемый код программы формируется из двух файлов: 


\begin{itemize}
\item 	detector.c -- прикладные функции для работы прибора ДЭПРОН


\item 	main.cpp -- инициализация контроллера прибора и функции обмена информацией с БИ. В процедуре main этого файла работает основной бесконечный цикл программы, в котором вызываются функции обмена информацией по каналу CAN и процедура Detectors\_Handling.


\end{itemize}
Работа прибора ДЭПРОН основана на прерываниях, которые обрабатываются по мере их поступления в процедуре Ext\_Interrupt, а ресурсоемкий разбор полученных данных и запуск АЦП происходят в процедуре Detectors\_Handling, которая отрабатывает постоянно. 





Блок схема работы процедуры Ext\_Interrupt, \colorbox[rgb]{1.000,0.000,0.000}{требует обновления!}





Таблица Распределение битов в регистре прерывания


\textbf{Номер бита, начиная с младшего (PB\#\#)Источник прерыванияПримечаниеОписание}0J4 \# 1SPI канал1J4 \# 3SPI канал2J4 \# 5SPI канал3J4 \# 7SPI канал4JP2 \# 5Нейтрон, 1-й счетчик5JP2 \# 11t-сигнал 1-го детекторавехний п-п детектор6JP2 \# 12Пустой7JP2 \# 8Пустой8JP2 \# 16Пустой9JP2 \# 15Пустой10JP2 \# 1Окончание сигнала S111JP2 \# 7Нейтрон, 2-й счетчик12JP2 \# 9t-сигнал 1-го детекторааппаратный счетчик13JP2 \# 13t-сигнал 2-го детекторанижний п-п детектор14JP2 \# 14Пустой15JP2 \# 10Пустой16JP \# 3Окончание сигнала S2электрически не подключенJP \# 2GNDJP \# 4GNDJP \# 6GND








\colorbox[rgb]{1.000,0.000,0.000}{ДОБАВИТЬ:}


Блок схема работы процедуры Detectors\_Handling





\subsection*{	\textbf{ }\textbf{КПА}}

Контрольно приемная аппаратура прибора ДЭПРОН используется для  проведения автономных испытаний прибора. КПА ДЭПРОН состоит из:


\begin{itemize}
\item 	Ноутбук (или другой персональный компьютер) с установленной операционной системой Windows XP и установленным специальным программным обеспечением (программой Depron Terminal), наличием порта RS232, либо дополнительно преобразователь интерфейсов USBRS232;


\item 	Блока питания, обеспечивающего измерение потребляемого тока нагрузки GwINSTEK GPS-4303;


\item 	Преобразователя интерфейсов USBRS232 (при отсутствии COM порта у ПК);


\item 	Комплекта соединительных кабелей 


\item 	Блока КП -- контрольно-приемного блока


\end{itemize}
 





Рис.1. Схема подключения КПА для проверки функционирования прибора ДЭПРОН.


Блок КП предназначен для подключения генератора и осциллографа к тестовым входам прибора ДЭПРОН, а также для контроля наличия рабочих напряжений в контурах прибора. Блок КП имеет 4 входных гнезда BNC промаркированных в соответствии с каналами прибора ДЭПРОН на которые передаются тестовые сигналы с генератора: 


\begin{itemize}
\item 	\colorbox[rgb]{1.000,0.000,0.000}{X1}


\item 	\colorbox[rgb]{1.000,0.000,0.000}{X2}\colorbox[rgb]{1.000,0.000,0.000}{\ldots{}}


\end{itemize}
На лицевой панели блока КП расположены 2 светодиодных индикатора. Подключение Блока КП к прибору ДЭПРОН происходит через тестовый вход XT3 (типа РС19).


\subsection*{	\textbf{ }\textbf{Градуировочные  характеристики прибора}}










\newpage



\section*{\textbf{Глава 3 Обработка информации с прибора}\textbf{ ДЭПРОН}}

Отладка программного обеспечения и проверка обработки сигналов от детекторов на этапе  конструкторских работ производилась подключением канала RS232 к последовательному (COM) порту персонального компьютера. Программное обеспечение контроллера формирует отладочные посылки и массивы тестовых данных и отправляет по интерфейсу USART, реализованному на всех использовавшихся контроллерах. Такой способ передачи тестовых данных выбран как максимально приближенный к условиям реального функционирования прибора. 


В некоторых случаях проверки текущих значений регистров производились прошивкой микроконтроллера и включением его в режиме отладки. Несмотря на то, что этот способ рекомендуется производителями микроконтроллеров как основной метод для разработчиков встраиваемых систем, он неприменим на самых критичных этапах разработки и отладки программного обеспечения, так как непредсказуемо влияет на скорость обработки прерываний и реальную тактовую частоту ядра микроконтроллера.


Отработка работы прибора в комплексе научной аппаратуры позволяет использовать штатный способ передачи информации по каналу CAN, в таком случае критерием работы прибора является выдача от БИ содержательных блоков информации с меткой, соответствующей прибору ДЭПРОН. 


\subsection*{	\textbf{Общая схема обработки и распределения потоков информации}\textbf{ }}




\subsection*{	\textbf{Программ}\textbf{ы}\textbf{ обработки }\textbf{данных }\textbf{прибора ДЭПРОН}}

Для обработки данных и отладки работы прибора ДЭПРОН были использованы специально разработанные программные средства. Поскольку отладка прибора ДЭПРОН производится подключением по каналу RS232 а при работе в штатном режиме передача данных ведется по каналу CAN, для взаимодействия с прибором были написаны две программы.


\subsubsection*{\textbf{Программа Depron}\textbf{ }\textbf{Terminal}}

Данная программа предназначена для отладки прибора во время лабораторных испытаний, проверки работоспособности прибора при приемо-сдаточных работах. Программа была написана в средстве разработки ПО Microsoft Visual Studio на языке c\# c использованием фрэймворка .NET3.5. Пользовательский интерфейс программы построен на основе WinForms, поэтому отличается консервативностью и достаточно низкими аппаратными требованиями.


Программа позволяет:


\begin{itemize}
\item 	Подключаться к прибору ДЭПРОН по каналу RS232 (с использованием COM порта)


\item 	Принимать и отображать тестовые данные сформированные прибором ДЭПРОН


\item 	Сохранять запись потока данных на жесткий диск ПК (в фоновом режиме и по запросу)


\item 	Открывать сохраненные данные с носителя информации


\item 	Посылать команды на прибор ДЭПРОН (в том числе с заданной периодичностью)


\end{itemize}
 Рисунок. Интерфейс программы \textbf{Depron Terminal}





Данная программа была использована как основа для разработки отладочной программы для дозиметрических блоков ДБ-8м. Основные принципы работы новой программы, названной \textbf{DB}\textbf{8}\textbf{m}\textbf{ }\textbf{Terminal}\textbf{,} были сохранены и она обеспечивает те же базовые функции. Дополнительно программа обеспечивает возможность накопления спектров энерговыделения по детекторам ДБ-8м и отображения их в графическом виде, в режиме реального времени. 





Рисунок. Интерфейс программы \textbf{DB}\textbf{8}\textbf{m}\textbf{ }\textbf{Terminal}


Графическое отображение спектров реализовано с использованием компонента ZedGraph. Введение такой возможности значительно ускорило калибровку и градуировку прибора на источниках радиационного излучения, так что может быть рекомендовано для программ аналогичной направленности.


\subsubsection*{\textbf{Программа DepronExplorerView}}

Данная программа предназначена для просмотра и обработки данных прибора полученных во время комплексных испытаний или во время штатной работы прибора. Аналогично Depron Terminal, данная программа была написана в средстве разработки ПО Microsoft Visual Studio на языке c\# c использованием фрэймворка .NET3.5. Пользовательский интерфейс программы построен на основе WPF.


На момент комплексных испытаний прибора ДЭПРОН программа DepronExplorerView позволяет отображать все типы бинарных данных, полученных от прибора ДЭПРОН, в таблично-текстовой форме и сохранять полученные данные в текстовые файлы. Для удобства использования и повышения скорости работы интерфейс программы выполнен в стиле файлового менеджера.





\newpage



\subsection*{	\textbf{Структура массивов (базы данных) результатов измерений}}

Результаты измерений прибора ДЭПРОН формируется в массивы информации размером 512 байт.


Каждое сообщение состоит из следующих полей:


-  начало сообщения;


-  категория;


-  длина сообщения;


-  данные;


Поле {``}начало сообщения'' содержит 2 байта:


-  байт DLE -- 11110000;


-  байт STX -- 11111111;


На момент написания в программе ДЭПРОН используются нестандартные значения для байт DLE и STX, поэтому во избежание путаницы в дальнейших версия ПО ДЭПРОН будут использоваться общепринятые значения этих байт.


Поле {``}категория'' состоит из одного байта (CAT). При обмене с БИ используются варианты сообщений: A, S, H, N. Коды сообщений соответствуют таблице ASCII: A - 01000001, S - 01010011, H - 01001000, N -- 01001110.


Поле {``}длина сообщения'' содержит 1 байт (LEN) по умолчанию передается {``}\ensuremath{\backslash}0'', что означает общую длину посылки 512 байт.


В ином случае значение длины равно общему числу байт сообщения, исключая поле {``}начало сообщения''.


Поле {``}данные'' (RECORD) содержит данные в соответствии с описанием передаваемых сообщений и их спецификацией.





Общая структура сообщений выглядит следующим образом:


{\small Начало сообщения (DLE,STX)Категория(CAT)Длина(LEN)Данные(RECORD)Метка 1Метка 22 байта2 байта508 байт}


\newpage



	


	


	


	


	


\subsection*{	\textbf{Содержание блоков данных ДЭПРОН}}

Прибор ДЭПРОН в процессе штатной работы формирует несколько типов массивов информации, которые соответствуют различным типам измерений:


\begin{itemize}
\item 	дозиметрические измерения потока ионизирующих излучений;


\item 	измерения спектров потока ионизирующих излучений;


\item 	запись данных высокоэнергетичных событий в детекторах;


\item 	измерение временного характера кратковременных нейтронных явлений;


\end{itemize}
Также прибор ДЭПРОН формирует ответ на пришедшую команду от БИ.





Типы массивов данных прибора ДЭПРОН:


\begin{itemize}
\item 	блок данных ДЭПРОН  A  		Collected dose and fluxes values


\item 	блок данных ДЭПРОН  S 		Energy deposition spectra


\item 	блок данных ДЭПРОН  H  		High Amplitude Data


\item 	блок данных ДЭПРОН  N		Neutron burst data


\item 	блок данных ДЭПРОН  Т		квитанция на полученную команду


\end{itemize}
\textbf{{\large \newpage
}}


\begin{flushleft}

\end{flushleft}


\subsubsection*{\textbf{Блок данных ДЭПРОН A}}




{\small Данные (RECORD)Время Аппаратный счетчик детектора }{\small 1}{\small Счет детектора }{\small 1}{\small Счет детектора 2Счет совпадениймесяцденьчасмин1 байт1 байт1 байт1 байт2 байта2 байта2 байта2 байта}


Продолжение:


{\small Данные (RECORD)Нейтронный счетчик 1Нейтронный счетчик 2Доза по первому детекторуДоза по второму детекторуДоза совпаденийМассивы секундной динамики2 байта2 байта}{\small 4}{\small  байта}{\small 4}{\small  байта}{\small 4}{\small  байта480 байт}





{\small Блок массивов секундной динамики содержат шестьдесят массивов, содержащих приращения значений счетчиков за секунду, сжатых алгоритмом логарифмического сжатия.}





{\small Массив секундной динамикиАппаратный счетчик детектора }{\small 1}{\small Счет детектора }{\small 1}{\small Счет детектора 2Счет совпадений}1 байт1 байт1 байт1 байт


Продолжение:


{\small Массив секундной динамикиНейтронный счетчик 1Нейтронный счетчик 2Доза по первому детекторуДоза по второму детектору}1 байт1 байт1 байт1 байт


\newpage






\subsubsection*{\textbf{Блок данных ДЭПРОН S}}




{\footnotesize Данные (RECORD)Время Спектр детектора 1Счетчик детектора 1Спектр детектора 2Счетчик детектора 2Спектр совпадений 1Счетчик совпаденийСпектр совпадениймесяцденьчасмин1 байт1 байт1 байт1 байт124 байта4 байта124 байта4 байта124 байта4 байта124 байта}


Каждый передаваемый спектр содержит число зарегистрированных импульсов, попадающих в соответствующий энергетический канал детектора. Количество энергетических диапазонов 62, верхние границы каналов выбраны с помощью алгоритма логарифмического преобразования номера канала.


\textbf{{\small Канал №Код АЦПКанал №Код АЦПКанал №Код АЦПКанал №Код АЦП}}{\small 216181603464050256032419176357045128164322019236768523072540212083783253332864822224388965435847562324039960553840864242564010245640969722528841115257460810802632042128058512011882735243140859563212962838444153660614413104294164516646166561411230448461792627168151203148047192063768016128325124820481714433576492304}


\begin{flushleft}

\end{flushleft}


Таким образом, массивы спектров состоят из 62 четырехбайтных целых значений, порядок которых соответствует приведенному набору каналов.


\begin{flushleft}
\textbf{\newpage
}
\end{flushleft}


\subsubsection*{\textbf{Блок данных ДЭПРОН H}}

\begin{flushleft}

\end{flushleft}


{\footnotesize Данные (RECORD)Время Индекс массива высоких амплитуд(по умолчанию значение 63)Массивы данных по высоким амплитудаммесяцдень1 байт1 байт2 байта63 массива  по 8 байт}


\begin{flushleft}
Структура записи в массивы данных по высоким амплитудам:
\end{flushleft}


{\footnotesize Массив данных по высоким амплитудам Код АЦП 1Код АЦП 2Время событияКод таймерасекундаминутачас2 байта2 байта1 байт1 байт1 байт1 байт}


\begin{flushleft}

\end{flushleft}


\begin{flushleft}
\newpage

\end{flushleft}


\subsubsection*{\textbf{Блок данных ДЭПРОН N}}

\begin{flushleft}

\end{flushleft}


{\footnotesize Данные (RECORD)Массивы данных по нейтронным всплескам127 массивов  по 4 байт}


\begin{flushleft}
Структура записи в массив данных по нейтронным всплескам:
\end{flushleft}


3231302928272625242322212019181716{\small Время дня в секундах}


\begin{flushleft}

\end{flushleft}


151413121110987654321{\small НДКоличество тиков таймера после предыдущего нейтронного импульса}


\begin{flushleft}

\end{flushleft}


\begin{flushleft}
НД -- номер сработавшего нейтронного детектора:
\end{flushleft}


\begin{flushleft}
01 -- срабатывание первого детектора
\end{flushleft}


\begin{flushleft}
10 -- срабатывание второго детектора
\end{flushleft}


\begin{flushleft}
11 -- срабатывание обоих детекторов
\end{flushleft}


\begin{flushleft}

\end{flushleft}


\subsubsection*{\textbf{\newpage
Блок данных ДЭПРОН Т}}

\textbf{\textcolor[rgb]{1.000,0.000,0.000}{Внимание}}\textcolor[rgb]{1.000,0.000,0.000}{: В общей структуре сообщения для данного блока данных не выдается длина сообщения (}\textcolor[rgb]{1.000,0.000,0.000}{LEN}\textcolor[rgb]{1.000,0.000,0.000}{)! Исправлено 27.02.2013 -- теперь выдается {`}\ensuremath{\backslash}0'.}


Данный блок данных генерируется в ответ на пришедшее по каналу CAN от БИ командное сообщение, либо по мере заполнения выходного буфера при работе в режиме отладки прибора ДЭПРОН.


\begin{center}
\textbf{Команды прибора ДЭПРОН}
\end{center}


Для управления работой прибора ДЭПРОН предусмотрено 6 типов команд:


\begin{flushleft}
	Сброс настроек к заводским параметрам
\end{flushleft}


\begin{flushleft}
	Увеличение временного диапазона для нейтронных последовательностей
\end{flushleft}


\begin{flushleft}
	Уменьшение временного диапазона для нейтронных последовательностей
\end{flushleft}


\begin{flushleft}
	Увеличение полосы фильтра шумов протонных каналов
\end{flushleft}


\begin{flushleft}
	Уменьшение полосы фильтра шумов протонных каналов
\end{flushleft}


\begin{flushleft}
	Увеличение интервала времени сглаживания
\end{flushleft}





\textbf{{\small №КомандаОписание команды}}{\small 1Clr}{\small сброс настроек к заводским параметрам2Tn+увеличение интервала между моментами регистрации нейтронов3Tn-уменьшение интервала между моментами регистрации нейтронов4Psnr+увеличение допустимого протонного фона5Psnr\_уменьшение допустимого протонного фона6Alpha+увеличение интервала времени сглаживания}


Ответный массив информации от прибора ДЭПРОН


{\small Данные (RECORD)Время Номер команды породившей ответD}{\small  }{\small tickСчетчик принятых командТекущий фон потока протонов (}{\small Pronon}{\small \_}{\small Fon}{\small )PsnrAlfaмесденьчасминсек1 байт1 байт1 байт1 байт1 байт4 байта}{\small 4}{\small  байта4 байта}{\small 4}{\small  байта}{\small 4}{\small  байта}{\small 4}{\small  байта}


\begin{flushleft}
Где:
\end{flushleft}


\begin{flushleft}
	D tick - интервал времени меньше которого считается, что идет один нейтронный всплеск, мсек/6
\end{flushleft}


	Psnr - максимальный уровень протонного фона, выше которого нейтронные всплески не регистрируются.


	Alfa - константа экспоненциального сглаживания для расчета фона протонов





\begin{center}
\textbf{Отладочные сообщения прибора ДЭПРОН}
\end{center}


Для проверки работы таймера высоких амплитуд и последовательности выполнения программного кода прибора ДЭПРОН существует возможность выдачи последовательностей измеренных временных промежутков, маркированных по названиям выполняемых блоков программного кода. Такая выдача происходит во время включения прибора ДЭПРОН в отладочном режиме, для этого необходима прошивка контроллера с объявленным макросом DEBUGTIME.


Измерение времени выполнения блоков происходит с помощью таймера прибора TC1. Информация записывается последовательно в блоки данных ДЭПРОН Т, каждый пакет имеет размер 4 байта, первые два байта отведены под идентификационные символы, последние два байта содержат накопленное значение таймера TC1, который настроен на работы с частотой 20 МГц.


{\small Данные (RECORD)Пакет таймера 1Пакет таймера 2\ldots{}\ldots{}\ldots{}\ldots{}Пакет таймера 2Символ 1Символ 2Значение таймера1 байт1 байт2 байт4 байт4 байт}





Таблица определения выполненных блоков по маркирующим символам пакета таймера.


\textbf{Символ 1Символ 2Выполненный блок}ADВыполнен блок External\_ADC\_Read\_DoubleE0x01Выполнен блок Ext\_Interrupt, вхождение блока P1E0x02Выполнен блок Ext\_Interrupt, вхождение блока P2E0x04Выполнен блок Ext\_Interrupt, вхождение блока N1E0x08Выполнен блок Ext\_Interrupt, вхождение блока N2E0x10Выполнен блок Ext\_Interrupt, вхождение блока T2D0Выполнен блок Detectors\_Handling, Detectors\_Flugs пустоеD1Выполнен блок Detectors\_Handling, до вызова External\_ADC\_Read\_DoubleD2Выполнен блок Detectors\_Handling, после вызова External\_ADC\_Read\_Double и до конца функцииN0Выполнен блок New\_Secunde\_Handler (вызов Data\_CAN\_Sending и Command\_Handler каждую секунду)N1Выполнен блок New\_Secunde\_Handler (сохранение текущей дозы каждую секунду)N2Выполнен блок New\_Secunde\_Handler (отправка накопленных за минуту данных и каждые пять минут спектра)





\newpage






\begin{flushleft}

\end{flushleft}


\subsection*{	\textbf{Периодичность выдачи массивов данных}}




\textbf{{\large Блок}}\textbf{{\large  }}\textbf{{\large данныхСодержаниеПериодичность}}{\small АВеличины поглощенной дозы и потоков частиц1 мин.}{\small S}{\small Спектр энерговыделения5 мин.}{\small H}{\small Данные о высокоэнергетичных событияхПо мере накопления данных}{\small N}{\small Данные по нейтронным вспышкамПо мере накопления данных, но не более 10 массивов в минуту}{\small T}{\small Квитанция на полученную командуПо мере поступления команд квитанция на полученную команду}{\small T[DEBUG]}{\small Данные по времени выполнения блоков программыПо мере заполнения буфера отправки сообщений прибора}





\begin{flushleft}

\end{flushleft}














\newpage



\section*{\textbf{Глава 4 Результаты}\textbf{ }}

\subsection*{	\textbf{Планетарное распределение дозы на высоте полета КА а также потоков нейтронов}}

\subsection*{	\textbf{Спектры ЛПЭ и распределение мощности эквивалентной дозы}\textbf{ }}

\section*{\textbf{Заключение}}

\section*{\textbf{Выводы}}

\section*{\textbf{Благодарности}}

Данная работа была бы невозможна без всемерной поддержки моей жены Брильковой Любови Святославовны.


\section*{\textbf{Список литературы}}

\newpage






\newpage



\section*{\textbf{Приложение 1. }\textbf{\colorbox[rgb]{1.000,1.000,0.000}{Габаритный чертеж прибора Дэпрон}}}










































\end{document}
