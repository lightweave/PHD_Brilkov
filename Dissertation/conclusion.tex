\chapter*{Заключение}						% Заголовок
\addcontentsline{toc}{chapter}{Заключение}	% Добавляем его в оглавление

%% Согласно ГОСТ Р 7.0.11-2011:
%% 5.3.3 В заключении диссертации излагают итоги выполненного исследования, рекомендации, перспективы дальнейшей разработки темы.
%% 9.2.3 В заключении автореферата диссертации излагают итоги данного исследования, рекомендации и перспективы дальнейшей разработки темы.
%% Поэтому имеет смысл сделать эту часть общей и загрузить из одного файла в автореферат и в диссертацию:

Основные результаты работы заключаются в следующем.
%% Согласно ГОСТ Р 7.0.11-2011:
%% 5.3.3 В заключении диссертации излагают итоги выполненного исследования, рекомендации, перспективы дальнейшей разработки темы.
%% 9.2.3 В заключении автореферата диссертации излагают итоги данного исследования, рекомендации и перспективы дальнейшей разработки темы.
\begin{enumerate}
  \item На основе анализа литературных данных о радиационной обстановке на низковысотных орбитах  и результатов космических экспериментов по изучению радиационных условий космического пространства разработаны требования к дозиметрической измерительной аппаратуре;
  \item Для выполнения поставленных задач был создан активный дозиметр нового типа с возможностью регистрации нейтронов тепловых энергий;
  \item Численные исследования показали, что 
  \item Приближенные оценки показали, что полупроводниковые детекторы прибора чувствительны к электронам энергий более 0,5 МэВ и протонам с энергиями более 5 МэВ;
  \item Математическое моделирование показало что максимум функции чувствительности нейтронный счетчиков соответствует энергии нейтронов 0,005 МэВ и 0,05 МэВ детекторов различной защищенности;
  
\end{enumerate}

\section{Благодарности}
Автор выражает благодарность научному руководителю Бенгину В.В. прошедшему огромный путь вместе с диссертантом, и поддерживавшим словом и делом каждый момент времени. 

Также Золотарев И.А. выражает благодарность Панасюку М.И., давшему возможность включиться в большой образовательный проект - создание спутника Ломоносов и разрабатывать аппаратуру радиационного мониторинга. Также автор благодарен обучению которое обеспечили  все участники коллаборации космического эксперимента:  И.В. Яшин,  В.Л. Петров,  А.М. Амелюшкин и другие.  Автор благодарен коллегам  лаборатории  О.Ю. Нечаеву и  Л.С. Братолюбовой за руководство и поддержку. Особую благодарность автор выражает  О.Ю. Нечаеву и  Л.А. Смирнову за то что они,  своими руками, создали надежный прибор, который лег в основу настоящей диссертационной работы.

Автор благодарен за всемерную поддержку при подготовке диссертации  Ю.И. Логачева,  Б.Ю. Южкова, Е.Е. Антоновой и  В.И. Оседло и всех сотрудников отдела ОКФИ НИИЯФ МГУ.  

В заключение хочется отметить что данная работа была бы невозможна без всемерной поддержки жены автора Золотаревой Любови Святославовны.

\begin{figure}
	\centering
	\includegraphics[width=0.7\linewidth]{images/collab}
	\caption{Коллаборация космического эксперимента Ломоносов}
	\label{fig:collab}
\end{figure}
