%% Согласно ГОСТ Р 7.0.11-2011:
%% 5.3.3 В заключении диссертации излагают итоги выполненного исследования, рекомендации, перспективы дальнейшей разработки темы.
%% 9.2.3 В заключении автореферата диссертации излагают итоги данного исследования, рекомендации и перспективы дальнейшей разработки темы.

В процессе разработки аппаратуры для радиационного мониторинга на борту космического аппарата были выделены основные требования к дозиметру, которые позволили создать прибор ДЭПРОН, удовлетворяющий и выполнению задачи мониторинга обстановки и исследовательским целям в космической дозиметрии. Для решения этой задачи был проведен анализ современной литературы по направлению космической дозиметрии. Были найдены аналоги создаваемого прибора и определен ряд критических параметров для такого типа приборов, в том числе энергетические диапазоны регистрируемых излучений, типы излучений и ориентировочные потоки в изучаемых областях космического пространства. Эти параметры и определили оптимальные размеры детекторов, их расположение и требуемое быстродействие встраиваемых вычислительных процессоров.

Для выполнения поставленных задач был создан активный дозиметр нового типа с возможностью регистрации нейтронов тепловых энергий.

Средняя доза на 

\section{Выводы}
\begin{enumerate}
	\item Систематизированы и обобщены характеристики радиационных условий на аналогичных орбитах (аппараты БИОН, Прогноз, Cluster, POES) для разработки программы эксперимента;
	\item Разработаны требования к бортовому дозиметру для нового пилотируемого транспортного корабля;
	\item Разработан прибор для дозиметрического мониторинга на борту космического аппарата <<Ломоносов>>;
	\item Подготовлен и проведен эксперимент с дозиметром на борту КА <<Ломоносов>>;
	\item Обработана полученная с прибора ДЭПРОН информация и проведен её анализ.	
\end{enumerate}

Дополнительные результаты, полученные в результате проведения эксперимента ДЭПРОН: 
\begin{enumerate}

  \item Приближенные оценки показали, что полупроводниковые детекторы прибора чувствительны к электронам энергий более 0,5 МэВ и протонам с энергиями более 5 МэВ;
  \item Математическое моделирование показало что максимум функции чувствительности нейтронный счетчиков соответствует энергии нейтронов 0,005 МэВ и 0,05 МэВ детекторов различной защищенности;
  \item Поглощенная доза за время проведения эксперимента достигла ... и ... для верхнего и нижнего детекторов.
  
\end{enumerate}
