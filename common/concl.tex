%% Согласно ГОСТ Р 7.0.11-2011:
%% 5.3.3 В заключении диссертации излагают итоги выполненного исследования, рекомендации, перспективы дальнейшей разработки темы.
%% 9.2.3 В заключении автореферата диссертации излагают итоги данного исследования, рекомендации и перспективы дальнейшей разработки темы.
\begin{enumerate}
  \item На основе анализа литературных данных о радиационной обстановке на низковысотных орбитах  и результатов космических экспериментов по изучению радиационных условий космического пространства разработаны требования к дозиметрической измерительной аппаратуре;
  \item Для выполнения поставленных задач был создан активный дозиметр нового типа с возможностью регистрации нейтронов тепловых энергий;
  \item Численные исследования показали, что 
  \item Приближенные оценки показали, что полупроводниковые детекторы прибора чувствительны к электронам энергий более 0,5 МэВ и протонам с энергиями более 5 МэВ;
  \item Математическое моделирование показало что максимум функции чувствительности нейтронный счетчиков соответствует энергии нейтронов 0,005 МэВ и 0,05 МэВ детекторов различной защищенности;
  
\end{enumerate}
