{\actuality}
 Актуальность работы обусловлена планами создания пилотируемого транспортного корабля нового поколения, работающего на высокоширотных и лунных орбитах. Проект транспортного корабля активно разрабатывается с 2010~г. и к настоящему времени начата работа по выпуску рабочей конструкторской документации на составные части корабля, в том числе и на дозиметр бортовой.
 
 Несмотря на непрерывный дозиметрический контроль всех российских космических миссий, начиная с первого полета человека в космос и заканчивая полетами экспедиций на МКС, не вызывает сомнений необходимость продолжения ряда исследований радиационной обстановки на каждом из пилотируемых и на значительной части беспилотных космических аппаратах.  
 
 Именно поэтому необходимо разработать приборы  для проведения непрерывного дозиметрического мониторинга области околоземного пространства, в которой планируется проведение перспективных пилотируемых полетов. \textit{Данная работа направлена на создание основ для осуществления такого мониторинга.}

 \aim\ данной работы является разработка методов исследования распределения мощности дозы космической радиации и создание на основе этих методов современных приборов, предназначенных для космических аппаратов, работающих на околоземных и лунных орбитах.
  

Для~достижения поставленной цели в работе необходимо было решить следующие  {\tasks}:
\begin{enumerate}
  \item Систематизировать и обобщить характеристик радиационных условий на аналогичных орбитах (аппараты БИОН, Прогноз, Cluster, POES) для разработки программы эксперимента;
  \item Разработать требования к бортовому дозиметру для нового пилотируемого транспортного корабля;
  \item Разработать прибор для дозиметрического мониторинга на борту космического аппарата <<Ломоносов>>;
  \item Участвовать в подготовке и проведении эксперимента с дозиметром на борту КА <<Ломоносов>>;
  \item Обработать полученную с прибора ДЭПРОН информацию и провести её анализ.
  
\end{enumerate}

\defpositions
\begin{enumerate}
  \item Разработан прибор для радиационного мониторинга на борту КА
  \item Подтверждены измерительные характеристики нового прибора
  \item Получены и обработаны дозиметрические экспериментальные данные
  \item Выделен вклад в дозы при пересечении различных областей космического пространства - внутреннего и внешнего радиационного пояса
\end{enumerate}

\novelty
\begin{enumerate}
  \item Впервые разработан исследовательский прибор, сочетающий в едином блоке дозиметр заряженных частиц и нейтронные детекторы;
  \item Было выполнено оригинальное моделирование спектрометрических свойств дозиметра протонов электронов и нейтронов;
  \item Было выполнено создание базы данных спутниковых измерений прибора Дэпрон за все время работы прибора.
\end{enumerate}

\influence\ определяется необходимостью контроля радиационных условий в целях обеспечения надежной работы аппаратуры на полярных спутниках. Прибор ДЭПРОН позволил провести полноценное исследование всех основных компонент радиационного излучения, вносящих вклад в поглощенную и эквивалентную дозы на борту космического аппарата. 

\reliability\ полученных результатов обеспечивается публичностью данных спутниковых измерений радиационных условий. Результаты находятся в соответствии с исследованиями, проведенными на других приборах RELEC и ELFIN эксперимента Ломоносов, а также на других полярных спутниках (POES). Результаты измерений сравнивались с измерениями на баллонах BARREL (запуски август 2016г).

\probation\
Основные результаты работы докладывались~на:
международной конференции COSPAR2014, конференции ОМУС-2015 (г. Дубна), нескольких конференциях ``Ломоносовские чтения'', рабочих совещаний коллаборации Lomonosov, а также ряде других конференций.

\contribution\ Автор принимал активное участие в проведении тепловых испытаний полупроводникового детектора с усилительным трактом прибора. Проведены проверки работоспособности усилительного тракта с радиоактивными источниками ОСГИ Cs137 и Co60. Автор участвовал в проведении испытаний детекторов тепловых нейтронов на лабораторном источнике нейтронов. 

Проведены работы по стыковке и согласованию платы цифровой обработки сигналов с аналоговыми усилительными трактами и дискриминирующими блоками прибора. Автором вместе с его научным руководителем написана программа на С++ для контроллера платы цифровой обработки сигналов. Для наземной отработки и испытаний написана программа для ПК на WinForms/С\#, позволяющая оперативно контролировать параметры работы прибора и выходные данные. 

Создана исчерпывающая модель дозиметра в системе Catia и подготовлена для использования в Монте-Карло моделировании. Написана программа на базе пакета Geant4 для математического моделирования характеристик прибора ДЭПРОН, а также программы для моделирования характеристик приборов ДБ-8 и других приборов для радиационных измерений. 

Автором подготовлены программы для анализа данных прибора ДЭПРОН, в том числе программы предобработки бинарных данных и программы визуализации данных детекторов прибора. 
Автором подготовлена методика точной привязки спутниковых данных с прибора ДЭПРОН ко всемирному времени. С использованием оригинальной методики получена полная база данных измерений прибора ДЭПРОН за все время работы прибора.

\publications\ Основные результаты по теме диссертации изложены в ХХ печатных изданиях~\cite{zolotarev2017numerical51590279,vlasova2017optimization27547274,vlasova2015operational11246447},
Х из которых изданы в журналах, рекомендованных ВАК~\cite{zolotarev2016chislennoe32150868}, 
ХХ --- в тезисах докладов~\cite{zolotarev2016modelirovanie36997161,grafodatsky2016development21020642, amelyushkin2015sozdanie10657329, angelopoulos2010university5851829, angelopoulos2011university1295357}.
